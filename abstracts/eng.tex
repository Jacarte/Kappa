\Wasm, now the fourth officially recognized web language, enables web browsers to port native applications for the Web. 
Importantly, \Wasm has evolved into an essential element for backend scenarios such as cloud computing and edge computing. 
Therefore, \Wasm finds use in a plethora of applications, including but not limited to, web browsers, blockchain, and cloud computing. 
Despite the emphasis on security since its design and specification, \Wasm remains susceptible to various forms of attacks, including memory corruption and side-channels. 
Furthermore, \Wasm has been manipulated to disseminate malware, particularly in cases of browser cryptojacking. 
Interestingly, the predictability of the \Wasm ecosystem, encompassing its consumers and hosted programs, is remarkably high. 
Such predictability can amplify the effects of vulnerabilities within these ecosystems. 
For instance, a defect in a web browser, triggered by a faulty \Wasm program, could potentially impact millions of users. 

This thesis aims to bolster the security within the \Wasm ecosystem through the introduction of Software Diversification methods and tools. 
Software Diversification is a strategy designed to augment the costs of exploiting vulnerabilities by making software unpredictable.
The unpredictability within ecosystems can be diminished by automatically generating various program variants. 
These variants strengthen observable properties that are typically used to launch attacks, and in many instances, can completely eliminate such vulnerabilities. 


This work introduces three tools: CROW, MEWE, and WASM-MUTATE. 
Each tool has been specifically designed to tackle a unique facet of Software Diversification. 
Furthermore, these tools complement each other. 
We present empirical evidence demonstrating the potential application of Software Diversification to \Wasm programs in two distinct ways: Offensive and Defensive Software Diversification. 
Our research into Offensive Software Diversification in \Wasm unveils potential paths for enhancing the detection of \Wasm malware. 
On the contrary, our experiments in Defensive Software Diversification show that \Wasm programs can be fortified against side-channel attacks, specifically the Spectre attack.


