\msection{Intrinsic properties of diversification}
\label{exploit:discussion}
In \autoref{exploit:defensive}, we have noted an increasing trend of exfiltration bandwidth in certain variants. 
\autoref{offensive_app} presents a similar case, indicating that without a clear objective in the diversification process, uncontrolled diversification can be counterproductive. 
This implies that not all transformations contribute equally to the diversification objectives of \Wasm.


\wrule{Preservation:}
Some transformations yield distinct \Wasm binaries, yet their JIT compilation produces identical machine code.
Non-preserved transformations undermine the effectiveness of diversification, as discussed in \autoref{discussion}.
Incorporating random \texttt{nop} operations directly into \Wasm, for instance, does not alter the final machine code because JIT compilers frequently eliminate these \texttt{nop} operations.
This phenomenon is also observed with transformations to the custom sections of \Wasm binaries.
Identical machine code, even when their \Wasm variants are different, can be detected by malware detectors.
On the other hand, side-channel attacks occur at the machine code level, making \Wasm variants preservation essential code for successful diversification.


\wrule{Dead code addition:} 
Transformed code may not always execute. 
For example, Software Diversification might generate dead code or introduce a new function that the original program does not execute. 
This is beneficial for static analysis, whether for avoiding reverse engineering or evading static malware detection. 
However, dynamic analysis tools can identify this type of variant. 
This reduces the effectiveness of evasion. 
Furthermore, the inclusion of non-executing dead code does not affect side-channels, thereby not strengthening against attacks.
 
% More fine grained
\wrule{Disrupting accurate timers:} Cache timing side-channel attacks, including for the four binaries analyzed in \autoref{exploit:defensive}, depend on precise timers to measure cache access times. 
Disrupting these timers can effectively neutralize the attack \cite{JStimers}. 
The \Wasm variants inherently adopts a similar approach, introducing perturbations in the timing steps of \Wasm variants in case they are malicious. 
This is illustrated in \autoref{example:timer} and \autoref{example:timer2}, where the former shows the original time measurement and the latter presents a variant with introduced operations.
Remarkably, WASM-MUTATE is particularly effective in disrupting cache access timers. 
By introducing additional instructions, the inherent randomness in the time measurement of a single or a few instructions is amplified, thereby reducing the timer's accuracy. 


\lstdefinestyle{watcode}{
  numbers=none,
  stepnumber=1,
  numbersep=10pt,
  tabsize=4,
  showspaces=false,
  breaklines=true, 
  showstringspaces=false,
    moredelim=**[is][{\btHL[fill=weborange!40]}]{`}{`},
    moredelim=**[is][{\btHL[fill=celadon!40]}]{!}{!}
}


   \begin{minipage}[b]{\linewidth}
    \lstset{
        language=WAT,
                        style=watcode,
        basicstyle=\footnotesize\ttfamily,
                        columns=fullflexible,
                        breaklines=true}
        
        \begin{lstlisting}[label=example:timer,caption={Wasm timer code.},frame=b, captionpos=b]{Name}
;; Code from original btb_breakout
...
(call $readTimer)
(set_local $end_time)
... access to mem
(i64.sub (get_local $end_time ) (get_local $start_time))
(set_local $duration)
...

        \end{lstlisting}
\end{minipage}


\begin{minipage}[b]{\linewidth}
    \lstset{
        language=WAT,
                        style=watcode,
        basicstyle=\footnotesize\ttfamily,
                        columns=fullflexible,
                        breaklines=true}
        
        \begin{lstlisting}[label=example:timer2,caption={\Wasm variant with more instructions added in between time measurement.},frame=b, captionpos=b]{Name}
;; Variant code
...
(call $readTimer)
(set_local $end_time)
!<inserted instructions>!
... access to mem
!<inserted instructions>!
(i64.sub (get_local $end_time ) (get_local $start_time))
(set_local $duration)
...
        \end{lstlisting}
\end{minipage}


\todo{Recheck term-citation padding here}

\wrule{Padding speculated instructions:} Additionally, CPUs have a limit on the number of instructions they can cache. 
Diversification injects instructions to potentially exceed this limit, effectively disabling the speculative execution of memory accesses. 
This approach is akin to padding \cite{padding}, as demonstrated in \autoref{example:padding} and \autoref{example:padding2}.
This padding disrupts the binary code's layout in memory, hindering the attacker's ability to initiate speculative execution. 
Even if speculative execution occurs, the memory access does not proceed as the attacker intended.


\lstdefinestyle{watcode}{
  numbers=none,
  stepnumber=1,
  numbersep=10pt,
  tabsize=4,
  showspaces=false,
  breaklines=true, 
  showstringspaces=false,
    moredelim=**[is][{\btHL[fill=weborange!40]}]{`}{`},
    moredelim=**[is][{\btHL[fill=celadon!40]}]{!}{!}
}


   \begin{minipage}[b]{0.8\linewidth}
    \lstset{
        language=WAT,
                        style=watcode,
        basicstyle=\footnotesize\ttfamily,
                        columns=fullflexible,
                        breaklines=true}
        
        \begin{lstlisting}[label=example:padding,caption={Two jump locations. The top one trains the branch predictor, the bottom one is the expected jump that exfiltrates the memory access.},frame=b, captionpos=b]{Name}
;; Code from original btb_breakout
...
;; train the code to jump here (index 1)
(i32.load (i32.const 2000))
(i32.store (i32.const 83)) ;; just prevent optimization
...
;; transiently jump here
(i32.load (i32.const 339968)) ;; S(83) is the secret
(i32.store (i32.const 83)) ;; just prevent optimization
        \end{lstlisting}
\end{minipage}


\begin{minipage}[b]{0.8\linewidth}
    \lstset{
        language=WAT,
                        style=watcode,
        basicstyle=\footnotesize\ttfamily,
                        columns=fullflexible,
                        breaklines=true}
        
        \begin{lstlisting}[label=example:padding2,caption={\Wasm variant with more instructions added indindinctly between jump places.},frame=b, captionpos=b]{Name}
;; Variant code
...
;; train the code to jump here (index 1)
!<inserted instructions>!
(i32.load (i32.const 2000))
!<inserted instructions>!
(i32.store (i32.const 83)) ;; just prevent optimization
...
;; transiently jump here
!<inserted instructions>!
(i32.load (i32.const 339968)) ;; "S"(83) is the secret
!<inserted instructions>!
(i32.store (i32.const 83)) ;; just prevent optimization
...
        \end{lstlisting}
\end{minipage}




