WebAssembly, nu det fjärde officiellt erkända webbspråket, gör det möjligt för webbläsare att portera nativa applikationer till webben. Dessutom har WebAssembly utvecklats till en väsentlig komponent för backend-scenarier såsom molntjänster och edge-tjänster. Därmed används WebAssembly i en mängd olika applikationer, däribland webbläsare, blockchain och molntjänster. Trots sitt fokus på säkerhet från dess design till dess specifikation är WebAssembly fortfarande mottagligt för olika former av attacker, såsom minneskorruption och sidokanalattacker. Dessutom har WebAssembly manipulerats för att sprida skadlig programvara, särskilt otillåten cryptobrytning i webbläsare.

Webbsideresurser, inklusive de som innehåller exekverbar WebAssembly, skickas i en modern digital kontext huvudsakligen från centraliserade datacenter. Tusentals edge-enheter, i samarbete med webbläsarklienter, kör miljontals identiska WebAssembly-instantieringar varje sekund. Detta fenomen skapar ett högst förutsägbart ekosystem, där potentiella angripare kan förutse beteenden antingen i webbläsare eller backend-noder. En sådan förutsägbarhet ökar potentialen för sårbarheter inom dessa ekosystem och öppnar dörren för sidkanal- och minnesattacker med stor påverkan. Till exempel kan en brist i en webbläsare, framkallad av ett defekt WebAssembly-program, ha potential att påverka miljontals användare.

Denna avhandling syftar till att stärka säkerheten inom WebAssembly-ekosystemet genom införandet av metoder och verktyg för mjukvarudiversifiering. Mjukvarudiversifiering är en strategi som är utformad för att öka kostnaderna för att exploatera sårbarheter genom att göra programvaran oförutsägbar. Förutsägbarheten inom ekosystem kan minskas genom att automatiskt generera olika programvaruvarianter. Dessa varianter förstärker observerbara egenskaper som vanligtvis används för att starta attacker och kan i många fall helt eliminera sådana sårbarheter.

Detta arbete introducerar tre verktyg: CROW, MEWE och WASM-MUTATE. Varje verktyg har utformats specifikt för att hantera en unik aspekt av mjukvarudiversifiering. Vi presenterar empiriska bevis som visar på potentialen för tillämpning av våra metoder för mjukvarudiversifiering av WebAssembly-program på två distinkta sätt: offensiv och defensiv mjukvarudiversifiering. Vår forskning om offensiv mjukvarudiversifiering i WebAssembly avslöjar potentiella vägar för att förbättra upptäckten av WebAssembly-malware. Å andra sidan visar våra experiment inom defensiv mjukvarudiversifiering att WebAssembly-program kan härdas mot sidokanalattacker, särskilt Spectre-attacken.

