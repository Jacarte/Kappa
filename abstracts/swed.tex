\Wasm, nu det fjärde officiellt erkända webspråket, gör det möjligt för webbläsare att portera nativa applikationer till webben. Dessutom har \Wasm utvecklats till en väsentlig komponent för backend-scenarier såsom molnberäkning och kantberäkning. Därmed används \Wasm i en mängd olika applikationer, inklusive men inte begränsat till webbläsare, blockchain och molnberäkning. Trots betoningen på säkerhet sedan dess design och specifikation är \Wasm fortfarande mottagligt för olika former av attacker, inklusive minneskorruption och sidkanaler. Dessutom har \Wasm manipulerats för att sprida skadlig programvara, särskilt vid fall av browser cryptojacking.

Webbsidresurser, inklusive de som innehåller WebAssembly-binärer, serveras huvudsakligen från centraliserade datacenter i den moderna digitala miljön. Tusentals kantenheter, i samarbete med webbläsarklienter, kör miljontals identiska \Wasm-instantieringar varje sekund. Denna fenomen skapar ett högt förutsägbart ekosystem, där potentiella angripare kan förutse beteendet antingen i webbläsare eller backend-noder. En sådan förutsägbarhet ökar potentialen för sårbarheter inom dessa ekosystem och öppnar dörren för högpåverkande sidkanal- och minnesattacker. Till exempel kan en brist i en webbläsare, framkallad av ett defekt \Wasm-program, ha potential att påverka miljontals användare.

Denna avhandling syftar till att stärka säkerheten inom \Wasm-ekosystemet genom införandet av metoder och verktyg för mjukvarudiversifiering. Mjukvarudiversifiering är en strategi som är utformad för att öka kostnaderna för att exploatera sårbarheter genom att göra programvaran oförutsägbar. Oförutsägbarheten inom ekosystem kan minskas genom att automatiskt generera olika programvaruvarianter. Dessa varianter förstärker observerbara egenskaper som vanligtvis används för att starta attacker och kan i många fall helt eliminera sådana sårbarheter.

Detta arbete introducerar tre verktyg: CROW, MEWE och WASM-MUTATE. Varje verktyg har utformats specifikt för att hantera en unik aspekt av mjukvarudiversifiering. Vi presenterar empiriska bevis som visar på potentialen för tillämpning av våra metoder för mjukvarudiversifiering på \Wasm-program på två distinkta sätt: Offensiv och Defensiv mjukvarudiversifiering. Vår forskning om Offensiv mjukvarudiversifiering i \Wasm avslöjar potentiella vägar för att förbättra upptäckten av \Wasm-malware. Å andra sidan visar våra experiment inom Defensiv mjukvarudiversifiering att \Wasm-program kan härdat mot sidkanalattacker, särskilt Spectre-attacken.