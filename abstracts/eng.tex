\Wasm, now the fourth officially recognized web language, enables web browsers to port native applications for the Web. 
Furthermore, \Wasm has evolved into an essential element for backend scenarios such as cloud computing and edge computing. 
Therefore, \Wasm finds use in a plethora of applications, including but not limited to, web browsers, blockchain, and cloud computing. 
Despite the emphasis on security since its design and specification, \Wasm remains susceptible to various forms of attacks, including memory corruption and side-channels. 
Furthermore, \Wasm has been manipulated to disseminate malware, particularly in cases of browser cryptojacking. 

Web page resources, including those containing WebAssembly binaries, are predominantly served from centralized data centers in the modern digital landscape. 
In conjunction with browser clients, thousands of edge devices operate millions of identical \Wasm instantiations every second. 
This phenomenon creates a highly predictable ecosystem, wherein potential attackers can anticipate behavior either in browsers or backend nodes. 
Such predictability escalates the potential impact of vulnerabilities within these ecosystems, paving the way for high-impact side-channel and memory attacks. 
For instance, a flaw in a web browser, triggered by a defective \Wasm program, holds the potential to affect millions of users.


This work aims to harden the security within the \Wasm ecosystem through the introduction of Software Diversification methods and tools. 
Software Diversification is a strategy designed to augment the costs of exploiting vulnerabilities by making software less predictable.
The unpredictability within ecosystems can be diminished by automatically generating different, yet functionally equivalent, program variants. 
These variants strengthen observable properties that are typically used to launch attacks, and in many instances, can completely eliminate such vulnerabilities. 


This work introduces three tools: CROW, MEWE, and WASM-MUTATE. 
Each tool has been specifically designed to tackle a unique facet of Software Diversification. 
We present empirical evidence demonstrating the potential application of our Software Diversification methods to \Wasm programs in two distinct ways: Offensive and Defensive Software Diversification. 
Our research into Offensive Software Diversification in \Wasm unveils potential paths for enhancing the detection of \Wasm malware. 
On the other hand, our experiments in Defensive Software Diversification show that \Wasm programs can be hardened against side-channel attacks, specifically the Spectre attack.


