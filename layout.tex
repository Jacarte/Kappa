\definecolor{bluekth}{rgb}{0.16, 0.42, 0.705}

% 0.16, 0.42, 0.705  107/255


% chapter tiltes formatting
\titleformat{\chapter}[display]
  {\LARGE}
  {\renewcommand{\thechapter}{{\color{gray}0}\arabic{chapter}\color{gray}\vspace{2.0ex}\titlerule}
  %\colorbox{blacm}
  {{\bfseries\fontsize{40}{50}\selectfont
  \thechapter\hspace*{-1.5cm}
    \parbox[t][1.2cm][t]{1cm}{%
      \centering\textcolor{black}  {}}}}}
  {-3ex}
  {%\color{black}\titlerule
  \vspace{-12.5ex}\filleft\parbox[b]{0.5\textwidth}{\begin{flushright}\MakeUppercase{#1}\\\end{flushright}}\hspace*{-0.08cm}\vspace{-0.5cm}}
  [\vspace{0.1ex}]
% chapter tiltes spacing
\titlespacing*{\chapter}{0pt}{10pt}{50pt}

% section tiltes formatting
\titleformat{\section}
  {\LARGE}{\MySecSquare\ \thesection}{1em}{#1}

\titleformat{name=\section,numberless}
  {\Large}{\MySecSquare}{1em}{#1}

% subsection tiltes formatting
\titleformat{\subsection}
  {\large}{\MySecSquare}{1em}{#1}

\titleformat{name=\subsection,numberless}
  {\large}{\hspace{0.25em}\MySecSquare\hspace{0.45em}\thesubsection}{1em}{#1}



% formatting for chapter entries in ToC  
%\titlecontents{chapter}
%  [1em]{}
%  {\tiny\thecontentslabel{2.3em}}
%  {\hspace*{-2.3em}}
%  {\hfill\contentspage}
  

% formatting for section entries in ToC  
\titlecontents{section}
  [3.0em]
  {\addvspace{3pt}}%
  {\normalsize\contentslabel{2.3em}}
  {}
  {\titlerule*[1pc]{.}\contentspage}


%\titlecontents{subsection}
%  [10.0em]
%  {\addvspace{3pt}}
%  {\normalsize\contentslabel{2.3em}}
%  {\hspace*{-2.3em}}
%  {\titlerule*[1pc]{.}\contentspage}


  
% Square to be used in itemize
\newcommand\MySquare{%
  \leavevmode\hbox to 1.2ex{\hss\vrule height .9ex width .7ex depth -.2ex\hss}}
% Square to be used in section titles
\newcommand\MySecSquare{%
  \leavevmode\hbox to 1.2ex{\hss\vrule height 1.3ex width 1.1ex depth -.2ex\hss}}


% First level of itemize uses a square
\renewcommand\labelitemi{\MySquare}

%\setcounter{tocdepth}{2}
%\setcounter{secnumdepth}{4}

%%% This is a patch for the correct numbering
\newcounter{patchino}


\newcommand{\msubsection}[1]{
  \stepcounter{patchino}
  \setcounter{subsection}{\thepatchino}
  \subsection{#1}
}


\newcommand{\msection}[1]{
  \setcounter{patchino}{0}
  \section{#1}
}