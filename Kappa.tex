\documentclass[openany,g5paper,electronic]{kthesis}

\usepackage[T1]{fontenc}
\usepackage{textcomp}
\usepackage{lmodern}
\usepackage[latin1,utf8]{inputenc}
\usepackage[swedish, english]{babel}
\usepackage{tocloft}
\usepackage{multirow}
\usepackage{adjustbox}
\usepackage{subcaption}
\usepackage{graphicx,booktabs}
\usepackage{float}
\usepackage{rotating}
\usepackage{amsthm}
\usepackage{adjustbox}
\usepackage{varwidth} %for the varwidth minipage environment
\usepackage[listings,skins]{tcolorbox}
\usepackage{csquotes}
\usepackage{makeidx} 
\usepackage{enumitem}
%\usepackage{graphics}
%\usepackage{amssymb}
%\usepackage{amsmath}
\usepackage{graphicx}
\usepackage{caption}
\usepackage{listings}
\usepackage{newfloat}
\usepackage[cmex10]{amsmath}
\usepackage{subcaption}
\usepackage[square,numbers]{natbib}
%\usepackage{color} 
%\usepackage{transparent} 
%\usepackage{bm} % bold math
%\usepackage{fmtcount}
%\usepackage{booktabs}
\usepackage{xspace}
\usepackage{tikz}
%\usepackage{algpseudocode}
%\usepackage{algorithm}
\usepackage{algorithm2e}
\usepackage{url}
\usepackage{xurl}
%\usepackage{cite}
\PassOptionsToPackage{hyphens}{url}\usepackage{hyperref}
\usepackage[singlelinecheck=off]{caption}
\hypersetup{
    colorlinks,
    citecolor=black,
    filecolor=black,
    linkcolor=black,
    urlcolor=black
}

% FONTS
\usepackage{lmodern}
\usepackage[T1]{fontenc}
\usepackage{xifthen}% provides \isempty test
\usepackage{xcolor}
\usepackage[explicit]{titlesec}
\usepackage{titletoc}
\usepackage{pdfpages}
\usepackage{comment}
\usepackage[
    left = \flqq{},% 
    right = \frqq{},% 
    leftsub = \flq{},% 
    rightsub = \frq{} %
]{dirtytalk}

% Layout
\definecolor{bluekth}{rgb}{0.16, 0.42, 0.705}

% 0.16, 0.42, 0.705  107/255


% chapter tiltes formatting
\titleformat{\chapter}[display]
  % Add the current date, REMOVE THIS in production
  {\LARGE}
  {\footnotetext[0]{\tiny Comp. time \today\ \currenttime\\}\renewcommand{\thechapter}{\arabic{chapter}\color{gray}\vspace{2.0ex}\titlerule}
  %\colorbox{blacm}
  {}}
  {-2ex}
  {\vspace{-1cm}\filleft\parbox[b]{0.5\textwidth}{\begin{flushright}{\bfseries\fontsize{65}{70}\selectfont
    \thechapter
      }
    \\\fontsize{18}{20}\selectfont\MakeUppercase{#1}\\\end{flushright}}\vspace{-0.5cm}}
  [\vspace{0.1ex}]
% chapter tiltes spacing
\titlespacing*{\chapter}{0pt}{10pt}{50pt}

% section tiltes formatting
\titleformat{\section}
  {\LARGE}{\thesection}{1em}{#1}

\titleformat{name=\section,numberless}
  {\Large}{\MySecSquare}{1em}{#1}

% subsection tiltes formatting
\titleformat{\subsection}
  {\LARGE}{}{1em}{\hspace{-0.5cm}\thesubsection\ #1}
  
\titleformat{name=\subsection,numberless}
  {\LARGE}{\hspace{0.25em}}{1em}{\thesubsection\ #1}

% formatting for section entries in ToC  
\titlecontents{section}
  [3.0em]
  {\addvspace{3.5pt}}%
  {\normalsize\contentslabel{2.3em}}
  {}
  {\titlerule*[1pc]{.}\contentspage}



  
% Square to be used in itemize
\newcommand\MySquare{%
  \leavevmode\hbox to 1.2ex{\hss\vrule height .9ex width .7ex depth -.2ex\hss}}
% Square to be used in section titles
\newcommand\MySecSquare{%
  \leavevmode\hbox to 1.2ex{\hss\vrule height 1.3ex width 1.1ex depth -.2ex\hss}}


% First level of itemize uses a square
\renewcommand\labelitemi{\MySquare}

%\setcounter{tocdepth}{2}
%\setcounter{secnumdepth}{4}


\usetikzlibrary{tikzmark,calc,decorations.pathreplacing}
\newcommand{\Depth}{2}
\newcommand{\Height}{2}
\newcommand{\Width}{2}

\renewcommand{\labelitemi}{\textcolor{IndianRed3}{\bfseries\textbullet}}


%% For autorefname
\addto\extrasenglish{%
  \renewcommand{\sectionautorefname}{Section}
  \renewcommand{\chapterautorefname}{Chapter}
  \renewcommand{\subsubsectionautorefname}{Subsection}
  \renewcommand{\subsectionautorefname}{Subsection}
  \renewcommand{\algorithmautorefname}{Algorithm}
}
\DeclareGraphicsExtensions{.pdf,.png,.jpg,.eps}

\DeclareFloatingEnvironment[fileext=frm,placement={tph},name=Listing]{code}
\captionsetup[lstlisting]{singlelinecheck=false, margin=0pt}


\newcommand{\termidx}[2][]{%
    \ifthenelse{\isempty{#1}}%
    {#2} % 
    {#1} %
    \index{#2}
  }
  \newcommand{\ie}{\textit{i.e.,}\xspace}
  \newcommand{\etal}{et al.\xspace}


\newcommand{\subscript}[2]{$#1 _ #2$}

\newcommand{\libsodiumfunctions}{869}
\newcommand{\qrcodefunctions}{1849}
\newcommand{\allmewefunctions}{\libsodiumfunctions + \qrcodefunctions}

% Execute a python script for small calculations
\newcommand{\pypy}[1]{\input{|python3 interpreter.py #1}}
\newcommand{\fromjson}[2]{\input{| jq -r '#2' '#1}}

\newcommand{\corpusrosetta}{Rosetta\xspace}
\newcommand{\corpussodium}{Libsodium\xspace}
\newcommand{\corpusqrcode}{QrCode\xspace}


\newcommand{\DTWStatic}{dt\_static\xspace}
\newcommand{\DTW}{TraceDiff\ }
\newcommand{\crow}{CROW\xspace}
\newcommand{\mewe}{MEWE\xspace}
\newcommand{\wmutate}{wasm-mutate\xspace}


\newcommand{\wasm}{Wasm\xspace}
\newcommand{\Wasm}{WebAssembly\xspace}


\usepackage{catchfile}
\newcommand{\getenv}[2][]{%
  \CatchFileEdef{\temp}{"|kpsewhich --var-value #2"}{\endlinechar=-1}%
  \if\relax\detokenize{#1}\relax\temp\else\let#1\temp\fi}




\newcommand{\wrule}[1]{
  \vspace{3mm}\noindent\textbf{#1}
}

\newcommand*\badge[1]{ \colorbox{red}{\color{white}#1}}
\newcommand*\badget[1]{\colorbox{red}{\color{white}#1}}
\newcommand*\badgeg[1]{\colorbox{green}{\color{white}#1}}


\makeatletter
\newenvironment{btHighlight}[1][]
{\begingroup\tikzset{bt@Highlight@par/.style={#1}}\begin{lrbox}{\@tempboxa}}
{\end{lrbox}\bt@HL@box[bt@Highlight@par]{\@tempboxa}\endgroup}


\definecolor{commentgreen}{RGB}{176, 176, 176}
\definecolor{rowcolor}{cmyk}{0,0.87,0.68,0.32}
\definecolor{rowcolor2}{cmyk}{ 20, 0, 37, 34}

\definecolor{eminence}{RGB}{108,48,130}
\definecolor{weborange}{RGB}{255,165,0}
\definecolor{frenchplum}{RGB}{129,20,82}
\definecolor{darkgreen}{RGB}{10, 92, 10}


\definecolor{celadon}{rgb}{0.67, 0.88, 0.69}
%\renewcommand{\blue}{}

\newcommand\btHL[1][]{%
  \begin{btHighlight}[#1]\bgroup\aftergroup\bt@HL@endenv%
}
\def\bt@HL@endenv{%
  \end{btHighlight}%   
  \egroup
}
\newcommand{\bt@HL@box}[2][]{%
  \tikz[#1]{%
    \pgfpathrectangle{\pgfpoint{1pt}{0pt}}{\pgfpoint{\wd #2}{\ht #2}}%
    \pgfusepath{use as bounding box}%
    \node[anchor=base west, fill=orange!30,outer sep=0pt,inner xsep=1pt, inner ysep=0pt, rounded corners=3pt, minimum height=\ht\strutbox+1pt,#1]{\raisebox{1pt}{\strut}\strut\usebox{#2}};
  }%
}
\makeatother

\makeatletter


\lstdefinelanguage{C}{
    otherkeywords={},
    morekeywords=[1]{const, int},
    morekeywords=[2]{0},
    morekeywords=[3]{add,const,mul,shl,get,rem_s,rem_u,ne,tee,sub,set,store},
    morekeywords=[4]{},
    morekeywords=[5]{global, get_global, mut, set_global, export, import,loop, memory, data, get_local,if, block,module, set_local,call,br_if,end, all,call_indirect,local,global,module, func, param, result, type},
    morekeywords=[6]{=,;},
    morekeywords=[7]{(,),[,],.},
    sensitive=false,
    morecomment=[l]{;},
    morecomment=[s]{;}{;},
    morestring=[b]",
    keywordstyle=[1]\color{eminence}\bfseries,
    keywordstyle=[3]\color{frenchplum},
    keywordstyle=[5]\color{darkgreen}\bfseries,
    commentstyle=\color{commentgreen}
}

\lstdefinelanguage{WAT}{
    otherkeywords={},
    morekeywords=[1]{i32,f32,i64,f64},
    morekeywords=[2]{0},
    morekeywords=[3]{add,const,mul,shl,get,rem_s,rem_u,ne,tee,sub,set,store},
    morekeywords=[4]{},
    morekeywords=[5]{global, get_global, mut, set_global, export, import,loop, memory, data, get_local,if, block,module, set_local,call,br_if,end, all,call_indirect,local,global,module, func, param, result, type},
    morekeywords=[6]{=,;},
    morekeywords=[7]{(,),[,],.},
    sensitive=false,
    morecomment=[l]{;},
    morecomment=[s]{;}{;},
    morestring=[b]",
    keywordstyle=[1]\color{eminence}\bfseries,
    keywordstyle=[3]\color{frenchplum},
    keywordstyle=[5]\color{darkgreen}\bfseries,
    commentstyle=\color{commentgreen}
}
\lstdefinelanguage{llvm}{
    morecomment = [l]{;},
    morestring=[b]", 
    sensitive = true,
    morekeywords=[2]{i32,f32,i64,f64},
    morekeywords=[3]{
        define, declare, global, constant,
        internal, external, private,
        linkonce, linkonce_odr, weak, weak_odr, appending,
        common, extern_weak,
        thread_local, dllimport, dllexport,
        hidden, protected, default,
        except, deplibs,
        volatile, fastcc, coldcc, cc, ccc,
        x86_stdcallcc, x86_fastcallcc,
        ptx_kernel, ptx_device,
        signext, zeroext, inreg, sret, nounwind, noreturn,
        nocapture, byval, nest, readnone, readonly, noalias, uwtable,
        inlinehint, noinline, alwaysinline, optsize, ssp, sspreq,
        noredzone, noimplicitfloat, naked, alignstack,
        module, asm, align, tail, to,
        addrspace, section, alias, sideeffect, c, gc,
        target, datalayout, triple,
        blockaddress
    },
    morekeywords=[4]{
        fadd, sub, fsub, mul, fmul,
        sdiv, udiv, fdiv, srem, urem, frem,
        and, or, xor,
        icmp, fcmp,
        eq, ne, ugt, uge, ult, ule, sgt, sge, slt, sle,
        oeq, ogt, oge, olt, ole, one, ord, ueq, ugt, uge,
        ult, ule, une, uno,
        nuw, nsw, exact, inbounds,
        phi, call, select, shl, lshr, ashr, va_arg,
        trunc, zext, sext,
        fptrunc, fpext, fptoui, fptosi, uitofp, sitofp,
        ptrtoint, inttoptr, bitcast,
        ret, br, indirectbr, switch, invoke, unwind, unreachable,
        malloc, alloca, free, load, store, getelementptr,
        extractelement, insertelement, shufflevector,
        extractvalue, insertvalue,
    },
    alsoletter={\%},
    keywordsprefix={\%},% All identifiers starting with '%' will be printed as first order keywords.
    keywordstyle=[1]\bfseries,% As mentioned above, these are the keywords starting with '%', like '%5'
    keywordstyle=[2]\color{eminence}\bfseries,
    keywordstyle=[3]\color{darkgreen}\bfseries,
    keywordstyle=[4]\color{frenchplum},
}
\makeatother

\newcommand{\todo}[1]{%
%\refstepcounter{todo}
\noindent\textbf{\badge{TODO}} {\color{red} #1}
%\addcontentsline{td}{todo}
%{\color{red}\thesection.\thetodo\xspace #1}
}

\newcommand{\done}[1]{%
\noindent\textbf{\badgeg{DONE}} {\color{green}#1}
}
\newcommand{\citationneeded}{
  \badget{[?]}
}

\newcommand*\step[1]{
\noindent\tikz[baseline=(char.base)]{
        \node[shape=circle,text=black,draw=black, fill=white,inner sep=1.2pt] (char) {#1};}}



\newtheorem{definition}{Definition}
\providecommand*{\definitionautorefname}{Definition}
\newtheorem{metric}{Metric}
\providecommand*{\metricautorefname}{Metric}



\newtheorem{property}{Property}
\providecommand*{\propertyautorefname}{Property}

\hyphenation{Web-Assembly}
\hyphenation{super-optimizers}
\hyphenation{super-optimize}

%\addcontentsline{td}{todo}
%{\color{red}\thesection.\thetodo\xspace Citation needed}}


\makeatletter
\lstset{
    %language=C,
    basicstyle=\ttfamily\footnotesize\lst@ifdisplaystyle\scriptsize\fi,
    escapeinside={\%*}{*)},
    captionpos=t
}
\makeatother


\lstdefinestyle{CStyle}{
  %numbers=none,
  stepnumber=1,
  numbersep=10pt,
  tabsize=4,
  showspaces=false,
  showstringspaces=false,
  basicstyle=\scriptsize\ttfamily,
  %moredelim=**[is][{\btHL[fill=black!10]}]{`}{`},
  moredelim=**[is][{\btHL[fill=celadon!40]}]{@}{@}
}

\lstdefinestyle{WATStyle}{
  numbers=left,
  stepnumber=1,
  numbersep=5pt,
  tabsize=4,
  showspaces=false,
  showstringspaces=false,
}

\lstdefinestyle{LLVMStyle}{
  numbers=none,
  stepnumber=0,
  numbersep=10pt,
  tabsize=4,
  showspaces=false,
  showstringspaces=true,
}



\getenv[\NOWIDOW]{NOWIDOW}
\ifthenelse{\equal{\NOWIDOW}{True}}%
    {
      \widowpenalties 1 10000
      \raggedbottom

      \setlength{\parskip}{20pt}

      \titlecontents{chapter}[0pc]
      {\addvspace{15pt}}%
      {\large\bfseries}
      {\large\bfseries}
      {\bfseries\hfill\large}%
      
      \titlecontents{subsection}
      [10.0em]{}
      {\normalsize\contentslabel{3.3em}}
      {\hspace*{-2.3em}}
      {\titlerule*[1pc]{ }}

      \titlecontents{section}
      [10.0em]{}
      {\normalsize\contentslabel{3.3em}}
      {\hspace*{-2.3em}}
      {\titlerule*[1pc]{ }}


      \renewcommand\labelitemi{}


        % section tiltes formatting
        \titleformat{\section}
        {\Large}{}{1em}{#1}
        \titleformat{name=\section,numberless}
        {\Large}{}{1em}{#1}

        % subsection tiltes formatting
        \titleformat{\subsection}
        {\Large}{}{1em}{#1}
        \titleformat{name=\subsection,numberless}
        {\Large}{}{1em}{#1}

        \titleformat{\chapter}[display]
        {\LARGE}
        {\renewcommand{\thechapter}{}
        %\colorbox{blacm}
        {{\bfseries\fontsize{30}{40}\selectfont
        \thechapter
          \parbox[c][1.2cm][c]{1cm}{%
            \centering\textcolor{black}  {}}}}}
        {-1ex}
        {%\color{black}\titlerule
        \vspace{-1.9ex}\filleft\MakeUppercase{#1}}
        [
        ]
    } % 
    {} %

%%%%%%%%%%%%%%%%%%%%%%%%%%%%%%%%%%%%%%%
%%%%%%%%% Document starts here %%%%%%%%
%%%%%%%%%%%%%%%%%%%%%%%%%%%%%%%%%%%%%%%
\begin{document}

%%%%%%%%%%%%%%%%%%%%%%%%%%%%%%%%%%%%%%%%%%
%%%%%% First and second pages %%%%%%%%%%%%
%%%%%%%%%%%%%%%%%%%%%%%%%%%%%%%%%%%%%%%%%%

%\title{Runtime randomization and perturbation for virtual machines.}
\title{ Artificial Software Diversification for WebAssembly }
% \subtitle{}
\author{Javier Cabrera-Arteaga }
\date{date}
\thesistype{Doctoral Thesis \\ Supervised by \\ Benoit Baudry and Martin Monperrus\\\\}
\imprint{Stockholm, Sweden, 2023}
\examen{Teknologie doktorexamen i elektroteknik}

%\disputationsdatum{fredagen den 18 januari 2023 klockan 14.00}
%\disputationslokal{Sal F3, Lindstedtsvägen 26, Kungliga Tekniska H\"{o}gskolan, Stockholm}

\publisher{Universitetsservice US AB}
\address{KTH Royal Institute of Technology \\School of Electrical Engineering and Computer Science\\ Division of Software and Computer Systems \\ SE-10044 Stockholm\\ Sweden}
\isbn{ISBN 100-}
%\issn{ISSN XXX} % No longer used at KTH
\trita{TRITA-EECS-AVL-2020:4}
\kthlogo{KTHLogo}

\pretolerance=10000
\tolerance=4000 
\emergencystretch=10pt
% Create title page using info above
\maketitle

\frontmatter % Pages i, ii, iii, iv, v etc.
%%%%%%%%%%%%%%%%%%%%%%%%%%%%%%%%%%%
%%%%%%%%%%% ABSTRACT %%%%%%%%%%%%%%
%%%%%%%%%%%%%%%%%%%%%%%%%%%%%%%%%%%
\begin{abstract}
\noindent \lipsum[1]

\end{abstract}

\bigskip \bigskip \bigskip \bigskip \bigskip

\setlength{\leftskip}{0.3 cm} \textbf {Keywords:} Lorem, Ipsum, Dolor, Sit, Amet

%%%%%%%%%%%%%%%%%%%%%%%%%%%%%%%%%%%%%%%%
%%%%%%%% SWEDISH ABSTRACT %%%%%%%%%%%%%%
%%%%%%%%%%%%%%%%%%%%%%%%%%%%%%%%%%%%%%%%
\newpage
\selectlanguage{swedish}
\begin{abstract}
\noindent \lipsum[1]
\end{abstract}
\selectlanguage{english}


%%%%%%%%%%%%%%%%%%%%%%%%%%%%%%%%%%%%%%%
%%%%%%% List of papers %%%%%%%%%%%%%%%%
%%%%%%%%%%%%%%%%%%%%%%%%%%%%%%%%%%%%%%%
\chapter{List of Papers}

%\let\thefootnote\relax\footnote{Paper I and III are published under license in \textit{Journal of X}}
\begin{enumerate}[I]
	\item \textbf{\textit{Superoptimization of WebAssembly Bytecode}} \\
		\textbf{Javier Cabrera-Arteaga},Shrinish Donde, Jian Gu, Orestis Floros, Lucas Satabin, Benoit Baudry, Martin Monperrus\\
		\textit{Conference Companion of the 4th International Conference on Art, Science, and Engineering of Programming (Programming 2021), MoreVMs} \\
        \url{https://doi.org/10.1145/3397537.3397567}
    \item \textbf{\textit{CROW: Code Diversification for WebAssembly}} \\
        \textbf{Javier Cabrera-Arteaga}, Orestis Floros, Oscar Vera-Pérez, Benoit Baudry, Martin Monperrus\\
        \textit{Network and Distributed System Security Symposium (NDSS 2021), MADWeb} \\
        \url{https://doi.org/10.14722/madweb.2021.23004}
    \item \textbf{\textit{Multi-Variant Execution at the Edge}} \\
        \textbf{Javier Cabrera-Arteaga},Pierre Laperdrix, Martin Monperrus, Benoit Baudry\\
        \textit{Conference on Computer and Communications Security (CCS 2022), Moving Target Defense (MTD)} \\
        \url{https://dl.acm.org/doi/abs/10.1145/3560828.3564007}
    \item \textbf{\textit{WebAssembly Diversification for Malware Evasion}} \\
        \textbf{Javier Cabrera-Arteaga},Tim Toady, Martin Monperrus, Benoit Baudry\\
        \textit{Computers \& Security, Volume 131, 2023} \\
        \url{https://www.sciencedirect.com/science/article/pii/S0167404823002067}
    \item \textbf{\textit{Wasm-mutate: Fast and Effective Binary Diversification for
    WebAssembly}} \\
        \textbf{Javier Cabrera-Arteaga}, Nick Fitzgerald, Martin Monperrus, Benoit Baudry\\
        %\textit{Conference Companion of the 4th International Conference on Art, Science, and Engineering of Programming (Programming 2021), MoreVMs} \\
        %\url{https://doi.org/10.1145/3397537.3397567}    
    \item \textbf{\textit{Scalable Comparison of JavaScript V8 Bytecode Traces}} \\
        \textbf{Javier Cabrera-Arteaga},Martin Monperrus, Benoit Baudry\\
        \textit{11th ACM SIGPLAN International Workshop on Virtual Machines and Intermediate Languages (SPLASH 2019)} \\
        \url{https://doi.org/10.1145/3358504.3361228}
\end{enumerate}

%\pagebreak

%%%%%%%%%%%%%%%%%%%%%%%% Papers NOT included in THESIS
%Other contributions by the author not included in the thesis.
%\begin{enumerate}[I]%
%	\setcounter{enumi}{1}
%	\item \textbf{\textit{Title of paper}} \\
%		\textbf{First author}, Second author \\
%		\textit{Journal (year)}
%\end{enumerate}


%%%%%%%%%%%%%%%%%%%%%%%%%%%%%%%%%%%%%%%
%%%%%%%% ACKNOWLEDGMENT %%%%%%%%%%%%%%%
%%%%%%%%%%%%%%%%%%%%%%%%%%%%%%%%%%%%%%%
\chapter{Acknowledgement}

\todo{W} \todo{O}

\todo{Jury}

\todo{C}

\todo{F}

%%%%%%%%%%%%%%%%%%%%%%%%%%%%%%%%
%%%%%%%% ACRONYMS %%%%%%%%%%%%%%
%%%%%%%%%%%%%%%%%%%%%%%%%%%%%%%%
\chapter{Acronyms}
List of commonly used acronyms: \\

\begin{tabular}{llll}
\textbf{AE}		&	Acronym examples \\

\end{tabular}

%\input{Content/Dedication}\clearpage

\mainmatter % Pages 1, 2, 3...
%%%%%%%%%%%%%%%%%%%%%%%%%%%%%%%%%%%%%%
%%%%%% TABLE OF CONTENTS %%%%%%%%%%%%%
%%%%%%%%%%%%%%%%%%%%%%%%%%%%%%%%%%%%%%
\tableofcontents
\part{Thesis}
\clearpage


%%%%%%%%%%%%%%%%%%%%%%%%%%%%%%%%%%%%%%%%%%%%%%%%%%%%%%
%%%%%%%%%%%%% CHAPTER 1: INTRODUCTION %%%%%%%%%%%%%%%%
%%%%%%%%%%%%%%%%%%%%%%%%%%%%%%%%%%%%%%%%%%%%%%%%%%%%%%
\chapter{Introduction}
\label{Intro}

\chapterprecishere{Jealous stepmother and sisters; magical aid by a beast; a marriage won by gifts magically provided; a bird revealing a secret; a recognition by aid of a ring; or show; or what not; a dénouement of punishment; a happy marriage - all those things, which in sequence, make up Cinderella, may and do occur in an incalculable number of other combinations.\par\raggedleft--- {\small\textup{MR. Cox} \textbf{1893}, Cinderella: Three hundred and forty-five variants \cite{cox1893cinderella}}}

\lettrine[lines=3]{T}{he} first web browser, Nexus, made its appearance in 1990 \cite{nexus}. 
At its inception, web browsing consisted solely of retrieving and displaying small, static textual web pages. 
In simpler terms, users could only read page content without any interactive components. 
However, the escalating computing power of devices, the proliferation of the internet, the valuation of internet based companies and the demand for more engaging user experiences birthed the concept of executing code in conjunction with web pages. 
In 1995, the Netscape browser revolutionized this concept by introducing JavaScript \cite{10.1007/978-3-642-14107-2_7}, a language that allowed code execution on the client-side.
Interactive web content immediately highlighted benefits: unlike classical native software, web applications do not require installation, are always up-to-date, and are accessible from any device with a web browser. 
Significantly, since the advent of Netscape, all browsers have offered JavaScript support. 
In the present day, the majority of web pages incorporate not only HTML but also JavaScript code, which is executed on client computers. 
Over the past several decades, web browsers have metamorphosed into JavaScript language virtual machines. 
They have evolved into intricate systems capable of running comprehensive applications, such as video and audio players, animation creators, and PDF document renderers, like the one displaying this document.

JavaScript is presently the most widely utilized scripting language in all contemporary web browsers \cite{mulazzani2013fast}.
However, it is not without limitations due to the inherent characteristics of the language.
For instance, each JavaScript engine necessitates the parsing and recompiling of the JavaScript code, resulting in substantial overhead.
In fact, just parsing and compiling JavaScript code consume the majority of the load times of websites \cite{clark}.
In addition to performance limitations, JavaScript also has security concerns \cite{10.1145/1190216.1190252}.
A notable example of this is the lack of memory isolation in JavaScript, which allows extraction of information from other processes \cite{10.1145/3412841.3442001}.
These issues led the Web Consortium (W3C) to standardize a bytecode for the web environment in 2015, which is the \Wasm(Wasm) language.
Hence, \Wasm became the fourth official language for the web.

\msection{\Wasm}

\Wasm was designed with a focus on speed, portability, self-containment, and security \cite{Haas_2017}.
It allows for all programs to be compiled ahead-of-time from source languages like C/C++ and Rust.
Third-party compilers create \Wasm binaries, potentially including optimizations, as in the case of LLVM.
The Wasm language defines its Instruction Set Architecture as an abstraction, similar to machine code instructions but independent of CPU architectures \cite{wasm_spec}.
This design allows web browsers to compile quickly to target architectures through a fast, one-to-one translation process.

The versatility of WebAssembly extends not just to web browsers, but to backend scenarios as well. 
Previous research has demonstrated the advantages of utilizing WebAssembly as an intermediary layer, noting improved startup times and more efficient memory usage when compared to containerization and virtualization \cite{pMendkiServerless, 1244493Jacobsson}. 
In response to these findings, the Bytecode Alliance introduced the WebAssembly System Interface (WASI) in 2019 \cite{bytecodealliance, WASI}. 
WASI facilitated the execution of \wasm\ with a POSIX system interface protocol, thus enabling the direct execution of Wasm in the operating system. 
To underline the significance of this development, consider the words of Solomon Hykes, the former CEO of Docker, in a tweet about \Wasm and WASI:

\vspace{5mm}
\begin{minipage}{0.95\linewidth}
	%\centering	
	\emph{If WASM+WASI existed in 2008, we wouldn't have needed to created Docker. That's how important it is. Webassembly on the server is the future of computing. A standardized system interface was the missing link. Let's hope WASI is up to the task! \cite{solomon}}

\end{minipage}
\vspace{5mm}

WebAssembly is praised for its security, especially for its design that prevents programs from accessing data beyond their own memory. 
However, there has been less focus on potential vulnerabilities and attacks within WebAssembly's own memory \cite{usenixWasm2020}.
In addition, WebAssembly binaries may be inherently vulnerable due to flaws in their source code \cite{DeRoover2022}. 
There are also significant risks from side-channel attacks, as demonstrated by various researchers \cite{ret2spec, 10.1145/3488932.3517411, Swivel}. 
These vulnerabilities are not limited to the web browser environment, as WebAssembly is also used in the backend.
Yet, existing WebAssembly research mostly reacts to existing vulnerabilities, leaving the potential for unidentified attacks. 
In this dissertation, we propose a proactive approach to enhance WebAssembly security through Software Diversification.

\msection{Software Diversification}


Software Diversification is the process of finding, creating, and deploying program variants of a given original program \cite{okhravi2013survey} for the sake of security.
Cohen \etal \cite{cohen1993operating} and Forrest \etal \cite{595185} pioneered this field by proposing software diversification through code transformations. 
They proposed to produce variants of programs while preserving their functionalities, aiming to mitigate vulnerabilities.
Since then, transformations aiming at reducing the predictability of observable behavior of programs have been proposed. For example, works on this direction proposed to diversify programs control flow \cite{davi2015isomeron}, instruction set \cite{barrantes2003randomized}, or the system calls they use \cite{Chew02mitigatingbuffer}. 
Several of these transformations can be combined to produce less predictable variants.

While previous works on software diversification demonstrated the removal of vulnerabilities, in all cases, it can be used as a preemptive solution.
For example, if a vulnerability is present in one program variant, discovering and disseminating it will not affect other variants.
Software diversification has been widely researched, yet, the field does not study its application to \Wasm.

This dissertation presents toolsets, approaches and methodologies designed to enhance \Wasm security proactively through Software Diversification.
First, Software Diversification could expand the capabilities of already settled \Wasm analysis tools by incorporating diversified program variants, making it more challenging for attackers to exploit any missed vulnerabilities.
Generated as proactive security, these diversified variants can simulate a broader set of real-world conditions, thereby making \Wasm analysis tools more accurate. 
Second, we noted that current solutions to mitigate side-channel attacks on \Wasm binaries are either specific to certain attacks or need the modification of runtimes.
Software Diversification could mitigate yet-unknown vulnerabilities on \Wasm binaries by generating diversified variants in a platform-agnostic manner.


\begin{comment}
\msection{Problem statement}

Web browsers and JavaScript have nearly three decades of development.
Since then, web browsers have grown, reaching several implementations \cite{grosskurth2005reference, GARCES2021111004}. 
Nevertheless, only Firefox, Chrome, Safari, and Edge dominate on user computers.
This means that, for 5 arbitrary devices (computers, tablets, smartphones) in a world of millions, at least two of them use the same web browser.
This highlights a software monoculture problem \cite{goth2003addressing}, as an ecosystem of machines running the exact same software. 
The monoculture concept is an analogy from biology \cite{lala2009monoculture}. 
It describes an ecosystem that faces extinction due the lack of diversity as all individuals share the exact same vulnerabilities.
In other words, many applications can crash due to a single shared vulnerability.

Nowadays, the serving of web pages, including WebAssembly code, is centralized and provided through main servers \cite{STRAC}.
Thus, a similar argument for software monoculture can be used for the Wasm code that is served to web browsers. 
Despite being designed for sandboxing and secure execution, \wasm\ is not exempt from vulnerabilities \cite{WebAssemblySecurity}.
For example, \wasm\ engines are vulnerable to speculative execution \cite{Narayan2021Swivel}, and C/C++ source code vulnerabilities might be ported to Wasm binaries \cite{DeRoover2022}.  
Therefore, the sharing of the \wasm\ code through web browsers, also includes \wasm\ vulnerabilities.


% Escalating
The software monoculture problem escalates if we consider the edge-cloud computing platforms and how they are adopting \Wasm to provide services, as we previously mentioned.
Concretely, along with browser clients, thousands of edge devices running Wasm as backend services might be affected due to vulnerabilities sharing.
This means that if one node in an edge network is vulnerable, all the others are vulnerable in the exact same manner as the same binary is replicated on each node. 
In other words, the same attacker payload would break all edge nodes at oncem a single distributed Wasm binary could unleash a worldwide catasthrophe.

One might think that the solution is to adopt more web browser and interpreters implementations. 
However, this is virtually impossible as 4 web browsers dominate the market and edge-cloud computing platforms are transparently executed in the backend.
Thus, a solution in this direction is doomed to fail. 
Another solution is to provide different WebAssembly codes.
For example, a different source code, yet equivalent, can be provided when a web page requests it \cite{CROW}. 
Consequently, millions of computers would execute different codes even though they use the same web browser.
This strategy is called Software Diversification.
\end{comment}

\section{List of contributions}

%\todo{The superoptimization paper is a foundation, yet, we do not generate diversity perse in that paper. Is there a need to mention the contributions explicitly in the table}

In the space of Software Diversification we make the following contributions.

\begin{enumerate}[label=\textbf{C\arabic*}, ref=C\arabic*]
	\item \label{methodcontrib} \wrule{Experimental contribution:} For each proposed technique we provide an artifact implementation and conduct experiments to assess its capabilities. The artifacts are publicly available. The protocols and results of assessing the artifacts provide guidance for future research.
	
	\item \label{therycontrib} \wrule{Theoretical contribution:} We propose a theoretical foundation in order to improve Software Diversification for \Wasm.
	
	\item \label{generationcontrib} \wrule{Diversity generation:} We generate \Wasm program variants.
	
	\item \label{defensivecontrib} \wrule{Defensive Diversification:} We assess how generated \Wasm program variants could be used for defensive purposes.
	
	\item \label{ofensivecontrib} \wrule{Offensive Diversification:} We assess how generated \Wasm program variants could be used for offensive purposes, yet improving security systems.
	
\end{enumerate}

\begin{table}
	\centering
	\begin{tabular}{l | l l l l }
		\multicolumn{1}{c|}{Contribution} & \multicolumn{4}{c}{Research papers} 
		\\
		&  \ref{crowpaper} & \ref{mewepaper} & \ref{wasmmutatepaper} & \ref{evasionpaper} \\
		\hline
		\ref{methodcontrib} Experimental contribution & \checkmark & \checkmark & \checkmark & \checkmark \\
		\ref{therycontrib} Theoretical contribution  & \checkmark &  &  \checkmark &  \\
		\ref{generationcontrib} Diversity generation   & \checkmark & \checkmark & \checkmark & \checkmark\\
		\ref{defensivecontrib} Defensive diversification   & \checkmark & \checkmark & \checkmark\\
		\ref{ofensivecontrib} Offensive diversification  & & & & \checkmark\\
	\end{tabular}
	\label{contribmap}
	\caption{}
\end{table}

\section{Summary of research papers}

This compilation thesis comprises the following research papers.

\begin{enumerate}[label={\textbf{P\arabic*}:}, ref={P\arabic*}]
	\item \label{crowpaper} \textbf{CROW: Code randomization for WebAssembly bytecode.} \\ 
	\textbf{Javier Cabrera-Arteaga}, Orestis Floros, Oscar Vera-Pérez, Benoit Baudry, Martin Monperrus\\
	\textit{ Measurements, Attacks, and Defenses for the Web (MADWeb 2021), 12 pages} \\
	\url{https://doi.org/10.14722/madweb.2021.23004}
	
	\wrule{Summary:} In this paper, we introduce the first entirely automated workflow for diversifying WebAssembly binaries. 
	We present CROW, an open-source tool that implements software diversification through enumerative synthesis. 
	We assess the capabilities of CROW and examine its application on real-world, security-sensitive programs.
	In general, CROW can create statically diverse variants. 
	Furthermore, we illustrate that the generated variants exhibit different behaviors at runtime.

	
	
	\item \label{mewepaper} \textbf{Multivariant execution at the Edge. } \\
	\textbf{Javier Cabrera-Arteaga},Pierre Laperdrix, Martin Monperrus, Benoit Baudry\\
    \textit{Moving Target Defense (MTD 2022), 12 pages} \\
    \url{https://dl.acm.org/doi/abs/10.1145/3560828.3564007} 
	\wrule{Summary:} 
	In this paper, we synthesize functionally equivalent variants of a deployed edge service. 
	These variants are encapsulated into a single multivariant WebAssembly binary, 
	When executing the service endpoint, a random variant is selected each time a function is invoked.
	Execution of these multivariant binaries occurs on the global edge platform provided by Fastly, as part of a research collaboration. 
	We demonstrate that these multivariant binaries present a diverse range of execution traces throughout the entire edge platform, distributed worldwide, effectively creating a moving target defense.

	
	\item \label{wasmmutatepaper}\textbf{Wasm-mutate: Fast and efficient software diversification for WebAssembly. }\\ 
	\textbf{Javier Cabrera-Arteaga}, Nicholas Fitzgerald, Martin Monperrus, Benoit Baudry\\
	\textit{Under review, 17 pages} \\
	\url{https://arxiv.org/pdf/2309.07638.pdf}    

	\wrule{Summary:}
	%%IMPROVED
	This paper introduces WASM-MUTATE, a compiler-agnostic WebAssembly diversification engine. 
	The engine is designed to swiftly generate semantically equivalent yet behaviorally diverse WebAssembly variants by leveraging an e-graph. 
	We show that WASM-MUTATE can generate tens of thousands of unique WebAssembly variants in mere minutes. 
	Importantly, WASM-MUTATE can safeguard WebAssembly binaries from timing side-channel attacks, such as Spectre.



	\item \label{evasionpaper} \textbf{WebAssembly Diversification for Malware evasion.} \\ 
	\textbf{Javier Cabrera-Arteaga},Tim Toady, Martin Monperrus, Benoit Baudry\\
	\textit{Computers \& Security, Volume 131, 2023, 17 pages} \\
	\wrule{Summary:}  \Wasm, while enhancing rich applications in browsers, also proves efficient in developing cryptojacking malware. 
	Protective measures against cryptomalware have not factored in the potential use of evasion techniques by attackers. 
	This paper delves into the potential of automatic binary diversification in aiding WebAssembly cryptojacking detectors' evasion. 
	We provide proof that our diversification tools can generate variants of WebAssembly cryptojacking that successfully evade VirusTotal and MINOS. 
	We further demonstrate that these generated variants introduce minimal performance overhead, thus verifying binary diversification as an effective evasion technique.

\end{enumerate}


%\todo{End up with the diagram locating our contributions in the landscape.}

\section*{Thesis layout}
This dissertation comprises two parts as a compilation thesis. 
Part one summarises the research papers included within, which is partially rooted in the author's licentiate thesis \cite{Lic}. 
Chapter 2 offers a background on \Wasm and the latest advancements in Software Diversification. 
Chapter 3 delves into our technical contributions. 
Chapter 4 exhibits two use cases applying our technical contributions. 
Chapter 5 concludes the thesis and outlines future research directions. 
The second part of this thesis incorporates all the papers discussed in part one.


\chapter{Background and state of the art}
\label{SOTA}


\section{\Wasm}

\section{Software diversification}

\section{Generating Software Diversification}

\subsection{Variants generation}

\subsection{Variants equivalence}

\section{Exploiting Software Diversification}

\section{Defensive Diversification}

\section{Offensive Diversification}

\section{Contributions of this thesis to Software Diversification for \Wasm}
\chapter{Automatic Software Diversification for WebAssembly}
\label{tech}

\todo{Start here. 4 pages each and 2 pages discussion. Target 20 pages.}

\section{Approach landscape}

\begin{figure}[h]
	\centering
	\includegraphics[width=1.0\textwidth]{figures/landscape.pdf}
	\caption{Approach landscape.}
	\label{fig:approach_landscape}
\end{figure}

\section{Compiler based approach}

\section{Binary based approach}


\section{Multivariant binaries}

%\subsection{wasm-mutate}

%\subsection{CROW}
%\todo{Do not mentioend superoptimizers. Talk in terms of SMT encoding.}

%\subsection{Constant inferring}

%\subsection{Disabling optimisations}

%\subsection{MEWE}


%\section{Approaches comparison}

%\section{Accompanying artifacts}
\chapter{Exploiting Software Diversification for \Wasm}
\label{exploit}

In this chapter we instantiate the usage of Software Diversification for offensive and defensive purposes.
We present two selected use cases that exploit Software Diversification through our technical contributions presented in \autoref{tech}.
The selected cases are representative of applications of Software Diversification for \Wasm in browsers and standalone engines.




\msection{Offensive Diversification: Malware evasion}
\label{offensive_app}

The primary malicious use of WebAssembly in browsers is cryptojacking \cite{musch2019new}. 
This is due to the essence of cryptojacking, the faster the mining, the better. 
Let us illustrate how a malicious \Wasm binary is involved into browser cryptojacking.
\autoref{fig:attack_crypto} illustrates a browser attack scenario:
a practical WebAssembly cryptojacking attack consists of three components: a WebAssembly binary, a JavaScript wrapper, and a backend cryptominer pool. 
The WebAssembly binary is responsible for executing the hash calculations, which consume significant computational resources. 
The JavaScript wrapper facilitates the communication between the WebAssembly binary and the cryptominer pool.

\begin{figure}[h]
    \centering
    \includegraphics[width=0.6\linewidth]{figures/attack_crypto.pdf}
    \caption{A remote mining pool server, a JavaScript wrapper and the \Wasm binary form the triad of a cryptojacking attack in browser clients.}
    \label{fig:attack_crypto}
\end{figure}

The aforementioned components require the following steps to succeed in cryptomining.
First, the victim visits a web page infected with the cryptojacking code. 
The web page establishes a channel to the cryptominer pool, which then assigns a hashing job to the infected browser. 
The WebAssembly cryptominer calculates thousands of hashes inside the browser. 
Once the malware server receives acceptable hashes, it is rewarded with cryptocurrencies for the mining. 
Then, the server assigns a new job, and the mining process starts over.

Both antivirus software and browsers have implemented measures to detect cryptojacking. For instance, Firefox employs deny lists to detect cryptomining activities \cite{firefoxcrypto}. 
The academic community has also contributed to the body of work on detecting or preventing WebAssembly-based cryptojacking, as outlined in \autoref{background:wasm:analysis}. 
However, malicious actors can employ evasion techniques to circumvent these detection mechanisms. 
Bhansali \etal are among the first who have investigated how WebAssembly cryptojacking could potentially evade detection \cite{10.1145/3507657.3528560}, highlighting the critical importance of this use case. 
The case illustrated in the subsequent sections uses Offensive Software Diversification for evading malware detection in \Wasm. 

\msubsection{Cryptojacking defense evasion}
\label{threat_model}


Considering the previous scenario, several techniques can be directly implemented in browsers to thwart cryptojacking by identifying the malicious \Wasm components. 
Such defense scenario is illustrated in \autoref{fig:threat_model}, where the \Wasm malicious binary is blocked in \step{3}.
The primary aim of our use case is to investigate the effectiveness of code diversification as a means to circumvent cryptojacking defenses. 
Specifically, we assess whether the following evasion workflow can successfully bypass existing security measures:

\begin{figure}
    \centering
    \includegraphics[width=0.8\linewidth]{figures/threat_model.pdf}
    \caption{Cryptojacking scenario in which the malware detection mechanism is bypassed by using an evasion technique.}
    \label{fig:threat_model}
\end{figure}


\begin{enumerate}
    
    \item The user loads a webpage infected with cryptojacking malware, which leverages network resources for execution—corresponding to \step{1} and \step{2} in \autoref{fig:threat_model}. 
    
    \item A malware detection mechanism (malware oracle) identifies and blocks malicious WebAssembly binaries at \step{3}. 
    For example, a network proxy could intercept and forward these resources to an external detection service via its API.
    
    \item Anticipating that a specific malware detection system is consistently used for defense, the attacker swiftly generates a variant of the WebAssembly cryptojacking malware designed to evade detection at \step{4}.
    
    \item The attacker delivers the modified binary instead of the original one \step{5}, which initiates the cryptojacking process and compromises the browser \step{6}. The detection method is not capable of detecting the malicious nature of the binary, and the attack is successful.
    
\end{enumerate}


\msubsection{Methodology}

Our aim is to empirically validate the workflow in \autoref{fig:threat_model}, i.e., using Offensive Software Diversification in evading malware detection systems.
To achieve this, we employ WASM-MUTATE for generating \Wasm malware variants.
In this study, we categorize malware detection mechanisms as malware oracles, which can be of two types: binary and numeric. 
A binary oracle provides a binary decision, labeling a \Wasm binary as either malicious or benign. 
In contrast, a numeric oracle returns a numerical value representing the confidence level of the detection.

\begin{definition}{Malware oracle:}
    \label{malware_oracle_def}
    A malware oracle is a detection mechanism that returns either a binary decision or a numerical value indicating the confidence level of the detection.
\end{definition}


We employ VirusTotal as a numeric oracle and MINOS \cite{MINOS} as a binary oracle. 
VirusTotal is an online service that analyzes files and returns a confidence score in the form of the number of antivirus that flag the input file as malware, thus qualifying as a numeric oracle. 
MINOS, on the other hand, converts \Wasm binaries into grayscale images and employs a convolutional neural network for classification. 
It returns a binary decision, making it a binary oracle.


We use the wasmbench dataset \cite{Hilbig2021AnES} to establish a ground truth. 
After running the wasmbench dataset through VirusTotal and MINOS, we identify 33 binaries that are: 1) flagged as malicious by at least one VirusTotal vendor and, 2) are also detected by MINOS.
Then, to simulate the evasion scenario in \autoref{fig:threat_model}, we use WASM-MUTATE to generate \Wasm binary variants to evade malware detection (\step{4} in \autoref{fig:threat_model}).
We use WASM-MUTATE in two configurations: feedback-guided and stochastic diversification.

\begin{definition}{Feedback-guided Diversification:}
    \label{controlled_def}
    In feedback-guided diversification, the transformation process of a \Wasm program is guided by a numeric oracle, which influences the probability of each transformation. For instance, WASM-MUTATE can be configured to apply transformations that minimize the oracle's confidence score. Note that feedback-guided diversification needs a numeric oracle.
\end{definition}


\begin{definition}{Stochastic Diversification:}
    \label{uncontrolled_def}
    Unlike feedback-guided diversification, in stochastic diversification, each transformation has an equal likelihood of being applied to the input \Wasm binary.
\end{definition}


Based on the two types of malware oracles and diversification configurations, we examine three scenarios:
1) VirusTotal with a feedback-guided diversification, 2) VirusTotal with an stochastic diversification, and 3) MINOS with a stochastic diversification.
Notice that, the fourth scenario with MINOS and a feedback-guided diversification is not feasible, as MINOS is a binary oracle and cannot provide the numerical values required for feedback-guided diversification.

Our evaluation focuses on two key metrics: the success rate of evading detection mechanisms in VirusTotal and MINOS across the 33 flagged binaries, and the correctness of the generated variants.

\begin{definition}{Evasion rate:} This measures the efficacy of WASM-MUTATE in bypassing malware detection systems. 
    For each flagged binary, we input it into WASM-MUTATE, configured with the selected oracle and diversification strategy. 
    We then iteratively apply transformations to the output from the preceding step. 
    This iterative process is halted either when the binary is no longer flagged by the oracle or when a maximum of 1000 stacked transformations have been applied (see \autoref{stack_transform}).
    This process is repeated with 10 random seeds per binary to simulate 10 different evasion experiments per binary.
\end{definition}

\begin{definition}{Correctness:} This verifies the functional equivalence of the variants generated by WASM-MUTATE compared to the original binary. 
    We execute the variants that entirely evade VirusTotal, using controlled and stochastic diversification configurations with WASM-MUTATE for both metrics. 
    Our selection is limited to variants that allow us to fully reproduce the three components displayed in \autoref{fig:attack_crypto}. 
    We then gather the hashes generated by the cryptojacking binaries and their generation speed, comparing these hashes with those from the original binary. 
    If the hashes match, and the variant executes without error, with the minerpool component validating the hash, we can consider the variant as functionally equivalent.
\end{definition}

\msubsection{Results}

In \autoref{offensive:results:fast}, we present a comprehensive summary of the evasion experiments presented in \cite{EVASION}, focusing on two oracles: VirusTotal and MINOS\cite{MINOS}. 
The table is organized into two main categories to separate the results for each malware oracle. 
For VirusTotal, we further subdivide the results based on the two diversification configurations we employ: stochastic and feedback-guided diversification. 
In these subsections, the columns indicate the number of VirusTotal vendors that flag the original binary as malware (\#D), the maximum number of successfully evaded detectors (Max. \#evaded), and the average number of transformations required (Mean \#trans.) for each sample. 
We highlight in bold text the values for which the stochastic diversification or feedback-guided diversification setups best, the lower, the better.
The MINOS section solely includes a column that specifies the number of transformations needed for complete evasion. 
The table has 33 + 1 rows, each representing a unique \Wasm malware study subject. 
The final row offers the median number of transformations required for evasion across our evaluated setups and oracles. 

\newcolumntype{t}{>{\columncolor{Gray}}r}
\begin{table}
    \footnotesize
    \centering
    \begin{tabular}{c | r | r l | r l | t }
        \hline
        & \multicolumn{5}{|c|}{VirusTotal} & MINOS\cite{MINOS} \\
        \hline
        Hash & \#D & \multicolumn{2}{c|}{Stochastic diversification} & \multicolumn{2}{c|}{Feedback-guided diversification} & \\
        \hline
        &  & Max. #evaded & Mean #trans. & Max. #evaded & Mean #trans. & Mean #trans. \\
        \hline\hlined
        47d29959 &                 31 &             \textbf{26} &     N/A & 19 & N/A  & 100  \\ 
        9d30e7f0 &                 30 &             \textbf{24}  &      N/A & 17 & N/A & 419  \\ 
        8ebf4e44 &                 26 &             \textbf{21} &     N/A  & 13 & N/A & 92 \\
        \hline
        c11d82d &                 20 &       20        &  \textbf{355} & 20 & 446 & 115 \\ 
        0d996462 &                 19 &     19    &  \textbf{401} & 19 & 697 & 24 \\ 
        a32a6f4b &                 18 &       18       &  635 & 18 & \textbf{625} & 1 \\
        
        
        fbdd1efa &                 18 &         18      &  \textbf{310} & 18 & 726 & 1 \\ 
        d2141ff2 &                  9 &          9      &  \textbf{461} & 9 & 781 & 81 \\ 
        aafff587 &                  6 &          6      &  484 & 6 & \textbf{331} & 1 \\
        
        
        046dc081 &                  6 &          6      &  404 & 6 & \textbf{159} & 33 \\ 
        643116ff &                  6 &          6      &  \textbf{144} & 6 & 436 & 47 \\ 
        15b86a25 &                  4 &          4      &  253 & 4 & \textbf{131} & 1 \\
        
        
        
        006b2fb6 &                  4 &           4     &  \textbf{282} & 4 & 380 & 1\\ 
        942be4f7 &                  4 &           4     &  200 & 4 & 200 & 29\\ 
        7c36f462 &                  4 &           4     &  236 & 4 & \textbf{221} & 85\\
        
        
        fb15929f &                  4 &            4    &  \textbf{297} & 4 & 475 & 1 \\ 
        24aae13a &                  4 &         4       &  \textbf{252 } & 4 & 401 & 980\\ 
        000415b2 &                  3 &         3       &  302 & 3 & \textbf{34} & 960 \\
        
        4cbdbbb1 &                  3 &          3      &  295 & 3 & \textbf{72} & 1\\ 
        65debcbe &                  2 &          2      &  131  & 2 & \textbf{33} & 38 \\ 
        59955b4c &                  2 &          2      &  130  & 2 & \textbf{33} & 38 \\
        
        
        89a3645c &                  2 &           2     &  431 & 2 & \textbf{107} & 108\\
        a74a7cb8 &                  2 &           2     &  124 & 2 & \textbf{33} & 38 \\
        119c53eb &                  2 &           2     &  104 & 2 & \textbf{18} & 1 \\
        
        089dd312 &                  2 &           2     &  153 & 2 & \textbf{123} & 68\\
        c1be4071 &                  2 &           2     &  130 & 2 & \textbf{33} & 38\\
        dceaf65b &                  2 &           2     &  140 & 2 & 132 & 66\\
        
        6b8c7899 &                  2 &            2    &  143 & 2 & \textbf{33} & 38 \\
        a27b45ef &                  2 &         2       &  145 & 2 & \textbf{33} & 33\\
        68ca7c0e &                  2 &         2       &  137  & 2 & \textbf{33} & 38\\
        
        f0b24409 &                  2 &         2       &  127  & 2 & \textbf{11} & 33 \\
        5bc53343 &                  2 &         2       &  118  & 2 & \textbf{33} & 33 \\
        e09c32c5 &                  1 &         1       &  \textbf{120}  & 1 & 488 & 15 \\
        \hline\hline
        Median &                         &         &      218  &   & 131 & 38
    \end{tabular}
    \caption{
        The table has two main categories for each malware oracle, corresponding to the two oracles we use: VirusTotal and MINOS. 
        For VirusTotal, divide the results based on the two diversification configurations: stochastic and feedback-guided diversification. 
        We provide columns that indicate the number of VirusTotal vendors that flag the original binary as malware (\#D), the maximum number of successfully evaded detectors (Max. \#evaded), and the average number of transformations required (Mean \#trans.) for each sample. 
        We highlight in bold text the values for which diversification setups are best, the lower, the better.
        The MINOS section includes a column that specifies the number of transformations needed for complete evasion. 
        The final row offers the median number of transformations required for evasion across our evaluated setups and oracles. 
    }
    \label{offensive:results:fast}
\end{table}

\begin{strategy}[Stochastic diversification to evade VirusTotal]
    \label{stochastic_div_vt}
    We execute a stochastic diversification with WASM-MUTATE, setting a limit of 1000 iterations for each binary. 
    In every iteration, we query VirusTotal to determine if the newly generated binary can elude detection. 
    We repeat this procedure with ten distinct seeds for each binary, replicating ten different evasion experiments. 
    As the stochastic diversification section of \autoref{offensive:results:fast} illustrates, we successfully produce variants that fully evade detection for 30 out of 33 binaries. 
    The average amount of iterations required to produce a variant that evades all detectors oscillates between 120 to 635 stacked transformations. 
    The mean number of iterations needed never exceeds 1000 stacked transformations. 
    However, three binaries remain detectable under the stochastic diversification setup. 
    In these instances, the algorithm fails to evade 5 out of 31, 6 out of 30, and 5 out of 26 detectors. 
    This shortfall can be attributed to the maximum number of iterations, 1000, that we employ in our experiments. 
    Increasing iterations further, however, seems unrealistic. 
    If certain transformations enlarge the binary size, a significantly large binary could become impractical due to bandwidth limitations. 
    In summary, stochastic diversification with WASM-MUTATE markedly reduces the detection rate by VirusTotal antivirus vendors for cryptojacking malware, achieving total evasion in 30 out of 33 (90\%) cases within the malware dataset. 
    %WASM-MUTATE proves capable of successfully evading detection systems in just a few minutes.    
\end{strategy}
    

\begin{strategy}[Feedback-guided diversification to evade VirusTotal]
    \label{guided_div_vt}
    stochastic diversification does not guide the diversification based on the number of evaded detectors, it is purely random, and has some drawbacks.
    For example, some transformations might suppress other transformations previously applied.
    We have observed that, by carefully selecting the order and type of transformations applied, it is possible to evade detection systems in fewer iterations.
    This can be appreciated in the results of the feedback-guided diversification part of \autoref{offensive:results:fast}.
    The feedback-guided diversification setup successfully generates variants that totally evade the detection for 30 out of 33 binaries, it is thus as good as the stochastic setup.
    Remarkably, for 21 binaries out of 30, feedback-guided needs only 40\% of the calls the stochastic diversification setup needs, demonstrating larger efficiency. 

    %In practice, a potential attacker may be limited by budget on the number of transformations applicable to the malware binary, e.g., the number of queries to VirusTotal. 
    %Additionally, the performance of the resulting binary benefits from this approach.

\end{strategy}
  
\begin{strategy}[Stochastic diversification to evade MINOS]
    \label{stochastic_div_minos}
    Relying exclusively on VirusTotal for detection could pose issues, particularly given the existence of specialized solutions for \Wasm, which differ from the general-purpose vendors within VirusTotal. 
    In \autoref{background:wasm:analysis} we highlight several examples of such solutions.
    Yet, for its simplicity, we extend this experiment by using MINOS\cite{MINOS}, an antivirus specifically designed for \Wasm. 
    The results of evading MINOS can be seen in the final column of \autoref{offensive:results:fast}.
    The bottom row of \autoref{offensive:results:fast} highlights that fewer iterations are required to evade MINOS than VirusTotal through WebAssembly diversification, indicating a greater ease in eluding MINOS.
    The stochastic diversification setup requires a median iteration count of 218 to evade VirusTotal. 
    In contrast, the feedback-guided diversification setup necessitates only 131 iterations. 
    Remarkably, a mere 38 iterations are needed for MINOS. 
    WASM-MUTATE evaded detection for 8 out of 33 binaries in a single iteration. 
    This result implies a vulnerability in the MINOS model to binary diversification.
\end{strategy}
    
%\msubsection{Efficiency and correctness results}
\vspace{5mm}
\begin{strategy}[\Wasm variants correctness]
    \label{evasion_impact}
    To evaluate the correctness of the malware variants created with WASM-MUTATE, we focused on six binaries that we could build and execute end-to-end, as these had all three components outlined in \autoref{fig:attack_crypto}. 
    We select only six binaries because the process of building and executing the binaries involves three components: the \Wasm binary, its JavaScript complement, and the miner pool. 
    These components were not found for the remaining 24 evaded binaries in the study subjects.
    For the six binaries, we then replace the original WebAssembly code with variants generated using VirusTotal as the malware oracle and WASM-MUTATE for both controlled and stochastic diversification configurations. 
    We then execute both the original and the generated variants. 
    We assess the correctness of the variants by examining the hashes they generate.
    Our findings show that all variants generated with WASM-MUTATE are correct, i.e., they generate the correct hashes and execute without error.
    Additionally, we found that 19\% of the generated variants surpassed the original cryptojacking binaries in performance.
\end{strategy}

% what's missing is to take a step back: what do those results validate? Wasm-mutate? the evasion process? both? what does this mean wrt to diversification in general?

%\todo{Do those results hold only for Spectre? do they generalize to other attacks? do you recommend binary diversification for any Wasm build pipeline as an hardening technique? how far are we from that?}

\begin{tcolorbox}[title=Reflection,boxrule=1pt,arc=.2em,boxsep=1.0mm]
    Our experiments conclusively demonstrate that WASM-MUTATE can effectively circumvent malware detection systems. 
    A possible key factor behind this is a misguided perception of resilience. 
    Malware detection is a well-known difficult problem \cite{cohen1987computer}. 
    Yet, prior research on static WebAssembly malware detection has shown an erroneous presumption: the existing of only metadata (\Wasm custom sections) obfuscation, or the complete absence of obfuscation techniques for \Wasm \cite{Minesweeper, MinerRay, SEISMIC, RAPID, MINOS}. 
    As explored in \autoref{sota:sw}, a software diversification engine can potentially function as an obfuscator. %%ORIGINAL
    The discussed use case partially demonstrates the assumption of non-existing obfuscators might be incorrect. 
    Consequently, our software diversification tools provide a viable solution for enhancing the accuracy of \Wasm malware detection systems.

    %Ultimately, our findings suggest that feedback-guided diversification is more efficient than stochastic diversification, requiring fewer transformations to avoid detection.    
\end{tcolorbox}


\begin{tcolorbox}[title=Contribution paper,boxrule=1pt,arc=.2em,boxsep=1.0mm]
    WASM-MUTATE generates correct and performant variants of WebAssembly cryptojacking that successfully evade malware detection.
    The case discussed in this section is fully detailed in Cabrera-Arteaga \etal "WebAssembly Diversification for Malware Evasion"
    \emph{at Computers \& Security, 2023}
    \url{https://www.sciencedirect.com/science/article/pii/S0167404823002067}. 
\end{tcolorbox}


% Efficiency
%We have found that 
%This improvement is attributed to WASM-MUTATE's ability to introduce code optimizations. 
%Additionally, debloating transformations, which eliminate unnecessary structures and dead code, resulted in a higher hash generation rate during the initial seconds of mining, likely due to faster compilation times. 
%This suggests that focused optimization serves as a valuable tool for evasion in browsers.
% The contrary case.
%On the contrary, 80\% of the generated variants are less efficient than the original binary, with the least efficient variant operating at only 20\% of the original hash generation rate. 
%This performance drop is primarily due to non-optimal transformations introduced by WASM-MUTATE. 
%Variants generated through stochastic diversification are generally slower.
%In summary, feedback-guided diversification yielded variants that evaded VirusTotal detection with minimal performance overhead—the worst-performing variant was only 1.93 times slower than the original.


\msection{Defensive Diversification: Speculative Side-channel protection}

As discussed in \autoref{background:wasm:ecosystems}, \Wasm is quickly becoming a cornerstone technology in backend systems. 
Leading companies like Cloudflare and Fastly are championing the integration of \Wasm into their edge computing platforms, thereby enabling developers to deploy applications that are both modular and securely sandboxed. 
These client-side \Wasm applications are generally architected as isolated, single-responsibility services, a model referred to as Function-as-a-Service (FaaS) \cite{pMendkiServerless, 1244493Jacobsson}. 
The operational flow of \Wasm binaries in FaaS platforms is illustrated in \autoref{fig:edge_model}.

\begin{figure}[h]
    \centering
    \includegraphics[width=0.8\linewidth]{figures/edge.pdf}
    \caption{\Wasm binaries on FaaS platforms. Developers can submit any \Wasm binary to the platform to be executed as a service in a sandboxed and isolated manner. Yet, \Wasm binaries are not immune to Spectre attacks.}
    \label{fig:edge_model}
\end{figure}


The fundamental advantage of using \Wasm in FaaS platforms lies in its ability to encapsulate thousands of client \Wasm binaries within a singular host process.
A developer could compile its source code into a \Wasm program suitable for the cloud platform and then submit it (\step{1} in \autoref{fig:edge_model}).
This host process is then disseminated across a network of servers and data centers (\step{2} in \autoref{fig:edge_model}). 
These platforms convert \Wasm programs into native code, which is subsequently executed in a sandboxed environment. 
Host processes can then instantiate new \Wasm sandboxes for each client function, executing them in response to specific user requests with nanosecond-level latency (\step{3} in \autoref{fig:edge_model}). 
This architecture inherently isolates \Wasm binary executions from each other as well as from the host process, enhancing security.

However, while \Wasm is engineered with a strong on security and isolation, it is not entirely immune to vulnerabilities such as Spectre attacks \cite{Spectre,Narayan2021Swivel} (\step{4} in \autoref{fig:edge_model}). 
In the sections that follow, we explore how software diversification techniques can be employed to fortify \Wasm binaries against such attacks. Dale ven

For an in-depth discussion on this topic, we direct the reader to our contribution \cite{wasmmutate}.

\msubsection{Threat model: speculative side-channel attacks}

To illustrate the threat model concerning \Wasm programs in FaaS platforms, consider the following scenarios. 
Developers, including potentially malicious actors, have the ability to submit any \Wasm binary to the FaaS platform. 
A malicious actor could then upload a \Wasm binary that, once compiled to native code, employs Spectre attacks to either leak sensitive information from the host process or violate Control Flow Integrity (CFI).
Furthermore, even if a submitted \Wasm binary is not intentionally malicious, it may still be vulnerable to Spectre attacks. 
For instance, a malicious actor could exploit this vulnerability by executing the susceptible binary through the FaaS service. 

Spectre attacks exploit hardware-based prediction mechanisms to trigger mispredictions, leading to the speculative execution of specific instruction sequences that are not part of the original, sequential execution flow. 
By taking advantage of this speculative execution, an attacker can potentially access sensitive information stored in the memory allocated to other \Wasm instance(including itself) or even the host process itself. 
This poses a significant risk, compromising both the security and integrity of the overall system.

Narayan and colleagues \cite{Narayan2021Swivel} have categorized potential Spectre attacks on \wasm binaries into three distinct types, each corresponding to a specific hardware predictor being exploited and a particular FaaS scenario: Branch Target Buffer Attacks,  Return Stack Buffer Attacks, and Pattern History Table Attacks defined as follows:

\begin{enumerate}
    \item The Spectre Branch Target Buffer (btb) attack exploits the branch target buffer by predicting the target of an indirect jump, thereby rerouting speculative control flow to an arbitrary target.
    \item  The Spectre Return Stack Buffer (rsb) attack exploits the return stack buffer that stores the locations of recently executed call instructions to predict the target of \texttt{ret} instructions.
    \item The Spectre Pattern History Table (pht) takes advantage of the pattern history table to anticipate the direction of a conditional branch during the ongoing evaluation of a condition.
\end{enumerate}


%\lipsum[1]

%\lipsum[1]

\msubsection{Methodology}

Our goal is to empirically validate that Software Diversification can effectively mitigate the risks associated with Spectre attacks in \Wasm binaries. 
The green-highlighted section in \autoref{fig:defense_model} illustrates how Software Diversification can be integrated into the FaaS platform workflow. 
The core idea is to generate unique and diverse \Wasm variants that can be randomized at the time of deployment. 
For this use case, we employ WASM-MUTATE as our tool for Software Diversification.

To empirically demonstrate that Software Diversification can indeed mitigate Spectre vulnerabilities, we reuse the \Wasm attack scenarios proposed by Narayan and colleagues in their work on Swivel \cite{Swivel}. 
Swivel is a compiler-based strategy designed to counteract Spectre attacks on \Wasm binaries by linearizing their control flow during machine code compilation. 
Our approach differs from theirs in that it is binary-based, compiler-agnostic, and platform-agnostic; we do not propose altering the deployment or toolchain of FaaS platforms. 
Although our experiments are conducted prior to submitting the \Wasm binary to the FaaS platform, we argue that \Wasm binary diversification could be implemented at any stage of the FaaS workflow.
The same argument holds by using any other diversification tool presented in this dissertation (see \autoref{tech}).


\begin{figure}[h]
    \centering
    \includegraphics[width=0.75\linewidth]{figures/edge_protected.pdf}
    \caption{Diversifying \Wasm binaries to mitigate Spectre attacks in FaaS platforms.}
    \label{fig:defense_model}
\end{figure}

\begin{table}
    \centering
    \begin{tabular}{l | l  }
        \hline
         Program &  Attack  \\
        \hline \hline
        btb\_breakout & Spectre branch target buffer (btb)  \\
        \hline
         btb\_leakage & Spectre branch target buffer(btb)  \\
        \hline
         ret2spec &  Spectre Return Stack Buffer (rsb)  \\
        \hline
        pht &  Spectre Pattern History Table (pht)  \\

%\end{adjustbox}
    \end{tabular}
    \caption{\Wasm program name and its respective attack.}
    \label{programs}
\end{table}

To measure the efficacy of WASM-MUTATE in mitigating Spectre, we diversify four \Wasm binaries proposed in the Swivel study. 
The names of these programs and the specific attacks we examine are available in \cite{programs}. 
For each of these four binaries, we generate up to 1000 random stacked transformations (see \autoref{stack_transform}) using 100 distinct seeds, resulting in a total of 100,000 variants for each original binary. 
At every 100th stacked transformation for each binary and seed, we assess the impact of diversification on the Spectre attacks by measuring the attack bandwidth for data exfiltration. 
This metric not only captures the success or failure of the attacks but also quantifies the extent to which data exfiltration is hindered. 
For example, a variant that still leaks data but does so at an impractically slow rate would be considered hardened against the attack.

\begin{definition}{Attack bandwidth:}\label{metric:ber}
    Given data $D=\{b_0, b1, ..., b_C\}$ being exfiltrated in time $T$ and $K = {k_1, k_2, ..., k_N}$ the collection of correct data bytes, the bandwidth metric is defined as:
    $$
        \frac{|b_i\text{ such that } b_i \in K|}{T}
    $$
\end{definition}


\msubsection{Results}


\autoref{attacks:impact:1} offers a graphical representation of \tool's influence on the Swivel original programs: btb\_breakout and btb\_leakage with the btb attack. 
The Y-axis represents the exfiltration bandwidth (see \autoref{metric:ber}). 
The bandwidth of the original binary under attack is marked as a blue dashed horizontal line.
In each plot, the variants are grouped in clusters of 100 stacked transformations. 
These are indicated by the green violinplots.

\begin{figure}[h]
    \centering
    \includegraphics[width=\linewidth]{plots/spectre/results.rq3.1.pdf}
    \caption{Impact of WASM-MUTATE over btb\_breakout and btb\_leakage binaries. The Y-axis denotes exfiltration bandwidth, with the original binary's bandwidth under attack highlighted by a blue marker and dashed line. Variants are clustered in groups of 100 stacked transformations, denoted by green violinplots. 
    Overall, for all 100000 variants generated out of each original program, 70\% have less data leakage bandwidth.
    After 200 stacked transformations, the exfiltration bandwidth drops to zero.
    }
  \label{attacks:impact:1}
\end{figure}

\wrule{Population Strength:} For the binaries btb\_breakout and btb\_leakage, \tool exhibits a high level of effectiveness, generating variants that leak less information than the original in 78\% and 70\% of instances, respectively.
For both programs, after applying 200 stacked transformations, the exfiltration bandwidth drops to zero.
This implies that \tool is capable of synthesizing variants that are entirely protected from the original attack.


\begin{tcolorbox}[title=Takeaway,boxrule=1pt,arc=.2em,boxsep=1.0mm]
    As indicated in \autoref{comp:table:tools}, generating a variant with 200 stacked transformations can be accomplished in just a matter of minutes.
    When scaled to the scope of a global FaaS platform, this means that a unique, fortified variant could be deployed for each machine and even for each fresh \Wasm spawned per user request.
\end{tcolorbox}






\begin{figure}[h]
    \centering    
    \includegraphics[width=\linewidth]{plots/spectre/results.rq3.2.pdf}
    \caption{Impact of WASM-MUTATE over ret2spec and pht binaries. The Y-axis denotes exfiltration bandwidth, with the original binary's bandwidth under attack highlighted by a blue marker and dashed line. Variants are clustered in groups of 100 stacked transformations, denoted by green violinplots. 
    Overall, for both programs approximately 70\% of the variants have less data leakage bandwidth.}
  \label{attacks:impact:2}
\end{figure}


% \todo{replace with violin plots}
As illustrated in \autoref{attacks:impact:2}, similarly to \autoref{attacks:impact:1}, WASM-MUTATE significantly impacts the programs ret2spec and pht when subjected to their respective attacks. 
In 76\% of instances for ret2spec and 71\% for pht, the generated variants demonstrated reduced attack bandwidth compared to the original binaries.
The plots reveal that a notable decrease in exfiltration bandwidth occurs after applying at least 100 stacked transformations. 
While both programs show signs of hardening through reduced attack bandwidth, this effect is not immediate and requires a substantial number of transformations to become effective. 
Additionally, the bandwidth distribution is more varied for these two programs compared to the two previous ones.
Our analysis suggests a correlation between the reduction in attack bandwidth and the complexity of the binary being diversified. 
Specifically, ret2spec and pht are substantially larger programs, containing over 300,000 instructions, compared to btb\_breakout and btb\_leakage, which have fewer than 800 instructions. 
Therefore, given that WASM-MUTATE performs incremental transformations, the probability of affecting critical components to hinder attacks decreases in larger binaries.

\wrule{Managed memory impact:} The success in diminishing exfiltration is explained by the fact that \tool synthesizes variants that effectively alter memory access patterns. 
We have identified four primary factors responsible for the divergence in memory accesses among \tool generated variants.
First, modifications to the binary layout—even those that don't affect executed code—inevitably alter memory accesses within the program's stack. 
Specifically, \tool generates variants that modify the return addresses of functions, which consequently leads to differences in execution flow and memory accesses.
Second, one of our rewriting rules incorporates artificial global values into \wasm binaries. 
Since these global variables are inherently manipulated via the stack, and given that the stack is located within linear memory, their access inevitably affects the managed memory (see \autoref{managed_unmanaged}).
Third, \tool injects 'phantom' instructions which don't aim to modify the outcome of a transformed function during execution. 
These intermediate calculations trigger the spill/reload component of the wasmtime compiler, varying spill and reload operations. 
In the context of limited physical resources, these operations temporarily store values in memory for later retrieval and use, thus creating diverse managed memory accesses(see the example at \autoref{custom}).
Finally, certain rewriting rules implemented by WASM-MUTATE replicate fragments of code, e.g., performing commutative operations. 
These code segments may contain memory accesses, and while neither the memory addresses nor their values change, the frequency of these operations does.

% More fine grained
\wrule{Disrupting accurate timers:} Cache timing side-channel attacks, including for the four binaries analyzed in this case study, depend on precise timers to measure cache access times. 
Disrupting these timers can effectively neutralize the attack. 
For example, in other contexts, Firefox employs a strategy to counter timing attacks by randomizing its built-in JavaScript timer \cite{10.1007/978-3-319-70972-7_13}. 
WASM-MUTATE inherently adopts a similar approach, introducing perturbations in the timing steps of \wasm variants in case they are malicious. 
This is illustrated in \autoref{example:timer} and \autoref{example:timer2}, where the former shows the original time measurement and the latter presents a variant with \tool-introduced operations.
WASM-MUTATE is particularly effective in disrupting cache access timers. 
By introducing additional instructions, the inherent randomness in the time measurement of a single or a few instructions is amplified, thereby reducing the timer's accuracy. 

\input{snippets/spectre/timer}

\wrule{Padding speculated instructions:} Additionally, CPUs have a limit on the number of instructions they can cache. 
WASM-MUTATE injects instructions to potentially exceed this limit, effectively disabling the speculative execution of memory accesses. 
This approach is akin to padding \cite{padding}, as demonstrated in \autoref{example:padding} and \autoref{example:padding2}.
This padding disrupts the binary code's layout in memory, hindering the attacker's ability to initiate speculative execution. 
Even if speculative execution occurs, the memory access does not proceed as the attacker intended.

\input{snippets/spectre/padding}

\wrule{Controlled vs Uncontrolled diversification:} We observed that the exfiltration bandwidth tends to increase in variants with only a few transformations. 
This suggests that not all transformations uniformly contribute to reducing data leakage. 
Several key factors contribute to this phenomenon.
First, as emphasized previously in \autoref{offensive_app}, uncontrolled diversification can be counterproductive if a specific objective, e.g., if a cost function, is not established at the beginning of the diversification process.
Second, while some transformations yield distinct \wasm binaries, their compilation produces identical machine code.
Transformations that are not preserved(see \autoref{discussion}) undermine the effectiveness of diversification.
For example, incorporating random \texttt{nop} operations directly into \wasm does not modify the final machine code as the \texttt{nop} operations are often removed by the compiler.
The same phenomenon is observed with transformations to custom sections of \Wasm binaries.
Additionally, it is important to note that transformed code doesn't always execute, i.e., \tool may generate dead code.





% \subsection{Deoptimization}

\begin{tcolorbox}[title=Contribution paper,boxrule=1pt,arc=.2em,boxsep=1.0mm]
    Software diversification crafts \Wasm binaries that are resilient to Spectre-like attacks. 
    By integrating a software diversification layer into \Wasm binaries deployed on Function-as-a-Service (FaaS) platforms, security can be significantly bolstered. 
    This approach allows for the deployment of unique and diversified \Wasm binaries, potentially utilizing a distinct variant for each cloud node, thereby enhancing the overall security.
    The case discussed in this section is fully detailed in Cabrera-Arteaga \etal "WASM-MUTATE: Fast and Effective Binary Diversification for WebAssembly"
    \emph{Under review}
    \url{https://arxiv.org/pdf/2309.07638.pdf}. 
\end{tcolorbox}





% \msection{Threats to validity}
\label{threats}

We discuss the threats to the validity of the two use cases presented in this chapter.
We separate the threats to validity into three main categories: internal, external, and construct validity.

\msubsection{Internal validity}


\msubsection{External validity}


\msubsection{Construct validity}


\subsection{Partial input/output validation}

% We need to talk about this because, we do this checking right noe and it is probably a reason for the low count of variants.
When \tool generates a variant, it can be executed to check the input/output equivalence.
If the variant has a \_start function, both binaries, the original and the variant can be initialized. 
If the state of the memory, the globals and the stack is the same after executing the \_start function, they are partially equivalent.
%This mechanismm is already implemented in the fuzzing campaign of wasmtime.

The \_start function is easier to execute given its signature.
It does not receive parameters.
Therefore, it can be executed directly.
Yet, since a \Wasm program might contain more than one function that could be indistinctly called with and arbitrary number of parameters, we are not able to validate the whole program.
Thus, we call the checking of the initialization of a \wasm variant, a partial validation.


\msection{Conclusions}

In this chapter we present two use of cases for Software Diversification in \Wasm.
We present a case of Offensive Software Diversification and a case of Defensive Software Diversification.
While the name of Offensive Software Diversification might seem counterproductive, it is important to note that the goal of this work is to highlight the potential of Software Diversification for \Wasm and to raise awareness of the security implications of \Wasm.
The latter highlights that Software Diversification can be used to protect \Wasm programs from malicious actors, specially from Spectre attacks.
In the next chapter we present the conclusions of this dissertation and future work.

\chapter{Conclusions and Future Work}
\label{results}

\chapterprecishere{\input{quotes/dn2.tex}\par\raggedleft--- {\small\textup{Donald Knuth}}}

\lettrine[lines=3]{T}{he} growing adoption of \Wasm requires efficent code analysis and hardening techniques.
This thesis contributes to this effort with a comprehensive set of methods and tools for Software Diversification in \Wasm.
We introduce three technical contributions in this dissertation: CROW, MEWE, and WASM-MUTATE.
Additionally, we present specific use cases for exploiting the diversification created for \Wasm programs.
In this chapter, we summarize the technical contributions of this dissertation, including an overview of the empirical findings of our research.
Finally, we discuss future research directions in \Wasm Software Diversification.

\section{Summary of technical contributions}


This thesis expands the field of Software Diversification within \Wasm by implementing two distinct methods: compiler-based and binary-based approaches. 
Taking source code and LLVM bitcode as input, the compiler-based method generates \Wasm variants.
It uses enumerative synthesis and SMT solvers to produce numerous functionally equivalent variants. 
Importantly, these generated variants can be converted into multivariant binaries, thus enabling execution path randomization. 
Our compiler-based approach specializes in producing high-preservation variants.
However, the use of SMT solvers for functional verification lowers the diversification speed when compared with the binary-based method. 
Furthermore, this method can only modify the code and function sections of \Wasm binaries.

On the other hand, the method based on binary utilizes random e-graph traversals to create variants. 
This approach eliminates the need for modifications to existing compilers, ensuring compatibility with all existing \Wasm binaries. 
Additionally, it offers a swift, efficient and novel method for generating variants through inexpensive random e-graph traversals. 
Consequently, our binary-based approach can produce variants at a scale at least one order of magnitude larger than our compiler-based approach. 
The binary-based method can generate variants by transforming any segment of the \wasm binary.
However, the preservation of the generated variants is lower than the compiler-based approach.

We have developed three open-source tools that are publicly accessible: CROW, MEWE, and WASM-MUTATE. 
CROW and MEWE utilize a compiler-based approach, whereas WASM-MUTATE employs a method based on binary. 
These tools automate the process of diversification, thereby increasing their practicality for deployment. 
At present, WASM-MUTATE is integrated in the wasmtime project\toolcite{https://github.com/bytecodealliance/wasm-tools} to improve testing. 
Our tools are complementary, providing combined utility. 
For instance, when the source code for a \Wasm binary is unavailable, WASM-MUTATE offers an efficient solution for the generation of code variants. 
On the other hand, CROW and MEWE are particularly suited for scenarios that require a high level of variant preservation.
Finally, one can use CROW and MEWE to generate a set of variants, which can then serve as rewriting rules for WASM-MUTATE. 
Moreover, when practitioners need to quickly generate variants, they could employ WASM-MUTATE, despite a potential decrease in the preservation of variants.


 

\section{Summary of empirical findings}

We demonstrate the practical application of Offensive Software Diversification in \Wasm.
In particular, we diversify 33 \Wasm cryptomalware automatically, generating numerous variants.
These variants successfully evade detection by state-of-the-art malware detection systems.
Our research confirms the existence of opportunities for the malware detection community to strengthen the automatic detection of cryptojacking WebAssembly malware.
Specifically, developers can improve the detection of \Wasm malware by using multiple malware oracles.
Additionally, these practitioners could employ feedback-guided diversification to identify specific transformations their implementation is susceptible to.
For instance, our study found that the addition of arbitrary custom sections to \Wasm binaries is a highly effective transformation for evading detection.
In practice, no \Wasm engine uses custom sections, so injecting them does not impact the performance of the \Wasm binary. 
This logic also applies to other transformations, such as adding unreachable code, another effective method for evading detection.

% Breaking side-channels
Moreover, our techniques enhance overall security from a Defensive Software Diversification perspective.
We facilitate the deployment of unique, diversified and hardened \Wasm binaries.
As previously demonstrated, \Wasm variants produced by our tools exhibit improved resistance to side-channel attacks.
Our tools generate variants by modifying malicious code patterns such as embedded timers used to conduct timing side-channel attacks.
Simultaneously, they can produce variants that introduce noise into the execution side-channels of the original program, concurrently altering the memory layout of the JITed code generated by the host engine.


To sum up, our methods remarkably generate tens of thousands of variants in minutes. 
The swift production of these variants is due to the rapid transformation of \Wasm binaries. 
This swift generation of variants is particularly advantageous in highly dynamic scenarios such as FaaS and CDN platforms. 
We have empirically tested the effectiveness of moving target defense techniques\cite{jackson2011compiler} on the Fastly edge computing platform. 
In this scenario, we incorporate multivariant executions\cite{MEWE}. 
Fastly can redeploy a \Wasm binary across its 73 datacenters worldwide in 13 seconds on average. 
This enables the practical deployment of a unique variant per node using our tools. 
However, a 13-second window may still pose a risk despite each node potentially hosting a distinct \Wasm variant. 
To mitigate this, we use multivariant binaries, invoking a unique variant with each execution. 
Our tools can generate dozens of unique variants every few seconds, each serving as a multivariant binary packaging thousands of other variants. 
This illustrates the real-world application of Defensive Software Diversification to a \Wasm standalone scenario.






%Implications of our implementations

\section{Future Work}

Along with this dissertation we have highlighted several open challenges related to Software Diversification in \Wasm.
These challenges open up several directions for future research.
In the following, we outline two concrete directions.



\begin{strategy}[Improving \Wasm malware detection via canonicalization]

    Malware detection is a well-known problem in the field of computer security, as outlined in works like Cohen's 1987 study on computer viruses \cite{cohen1987computer}. 
    This issue is exacerbated in environments where predictability is high and malware is expected to be replicated identically across multiple victims. 
    In such scenarios, attackers can exploit this predictability to their advantage. 
    For example, malicious actors could craft functionally equivalent malware to evade detection by malware detection systems.
    Indeed, our research has shown that employing Software Diversification can be an effective method for evading malware detection systems. 
    This technique involves creating varied versions of a program, thereby reducing its predictability and making detection more challenging. 
   

    In response to this challenge, our future research focuses on the potential effectiveness of program normalization in enhancing the accuracy of malware detection systems. 
    This approach involves transforming a program into a standardized, or "canonical," form before comparing it against a known dataset of malware signatures. 
    By doing this, we aim to rapidly and efficiently identify malicious programs. A key strategy we explore is the pre-compiling of \Wasm binaries, which can be done at a minimal cost. 
    For example, a Wasm binary might first be JIT compiled to machine code. 
    This step effectively reduces the number of malware variants that need to be considered, making it easier for classifiers to identify malicious software.


    However, this method is not without its challenges. 
    It relies heavily on the degree to which malware variants can be normalized.
    If a malware variant is highly preserved, meaning it maintains its original form and structure even after compiling to machine code, it might be difficult to normalize and subsequently detect. 
    This limitation suggests that while program normalization can significantly improve the efficiency and precision of malware detection systems, it is not a foolproof solution. 
    The ability to detect highly preserved malware variants remains an area for further research.
    

\end{strategy}


    
\begin{strategy}[Feedback-guided Diversification]
As presented in \autoref{exploit}, feedback-guided diversification can facilitate the identification of specific transformations aiding particular objectives such as malware evasion and side-channel protection. 
On the contrary, stochastic diversification may generate variants that do not align with the specific objective. 
For instance, our approaches have shown less impact on ret2spec and pht side-channel attacks compared to btb attacks when using stochastic diversification.
We can conceptualize this problem as an issue of search space exploration. 
By dividing the diversification search space, we can more efficiently focus the diversification process. 
The main benefit of this division is a narrowed search space, which can speed up and refine the process of diversification. 


We intend to research into the previously mentioned concept. 
Our strategy could potentially incorporate feedback-directed methods that rely on specific code patterns, such as the disruption of embedded timers, to combat these problems. 
This method might also be modified to lessen the impact of particular types of side-channel attacks, such as port contention \cite{10.1145/3488932.3517411}. 
For instance, one might count the number of instructions that result in port contention and use this data to choose or eliminate variants with a high or low count of instructions leading to port contention.



\begin{comment}%%ORIGINAL


We utilize \ref{rq:performance} to gauge the performance implications of the variants generated by \tool. 
A collection of 134 real-world programs is employed as our foundation for this task. 
The performance evaluation of the generated variants takes into account the \Wasm program size, wasmtime-compiled binary size, compilation duration, and the execution time of each program and its variants. 
For each program, we create a maximum of 50 unique variants, each containing 1000 stacked transformations diversified with 50 distinct seeds.
In conclusion, we assess these parameters for a total of 6834 programs, calculated as 134 plus $134\times50$.




\end{comment}

\end{strategy}
%\input{chapters/Chapter6}

%------------------------------------------------
% CREATING THE BIBLIOGRAPHY
%------------------------------------------------
\bibliographystyle{ieeetr}
\renewcommand{\bibname}{References}% changes default name Bibliography to References
\bibliography{Kappa} % References file

%--------------------------------------------------
%\addcontentsline{toc}{chapter}{Appended papers} % for template purposes
\part{Included papers}

% Each chapter is a paper
% chapter tiltes formatting
\titleformat{\chapter}[display]
  {\Large}
  {\renewcommand{\thechapter}{{\color{gray}P}\arabic{chapter}}\hspace*{0.5em}
  %\colorbox{blacm}
  {}}
  {-1ex}
  {%\color{black}\titlerule
  \vspace{-0.5cm}\huge\MakeUppercase{#1}}
  [\vspace{.0ex}
  \color{black}
  \titlerule
  ]
% chapter tiltes spacing
\titlespacing*{\chapter}{0pt}{20pt}{80pt}

\renewcommand{\thechapter}{}\hspace*{-1em}
  %\colorbox{blacm}
  {}
  \titlecontents{chapter}
  [-3.9pc]
  {{\addvspace{8pt}}}%
  {\normalsize}
  {\normalsize}
  {\hfill\large\contentspage}%

% \setcounter{chapter}{0}
\getenv[\ADDCONTRIB]{ADDCONTRIB}

\chapter{WebAssembly Diversification for Malware Evasion}

\textbf{Javier Cabrera-Arteaga}, Tim Toady, Martin Monperrus, Benoit Baudry\\
\emph{Computers \& Security, Volume 131, 2023}\\\\
\url{https://www.sciencedirect.com/science/article/pii/S0167404823002067}\\\\

\ifthenelse{\equal{\ADDCONTRIB}{True}}%
    {\includepdf[pages=1-16]{papers/evasion.pdf}} % 
    {} %
    

\chapter{\mbox{WASM-MUTATE}:~Fast~and~Effective~Binary~Diversification~for~\mbox{WebAssembly}}
\textbf{Javier Cabrera-Arteaga}, Nick Fitzgerald, Martin Monperrus, Benoit Baudry\\
\emph{Submitted to Computers \& Security, 2024}\\\\
\url{https://www.sciencedirect.com/science/article/pii/S0167404824000324}\\\\

\ifthenelse{\equal{\ADDCONTRIB}{True}}%
    {\includepdf[pages=1-20]{papers/wasm-mutate-v2.pdf}} % 
    {} %


% Add paper here
\chapter{CROW: Code Diversification for WebAssembly}

\textbf{Javier Cabrera-Arteaga}, Orestis Floros, Oscar Vera-Pérez, Benoit Baudry, Martin Monperrus\\
\emph{Network and Distributed System Security Symposium (NDSS 2021), Workshop on Measurements, Attacks, and Defenses for the Web}\\\\
\url{https://doi.org/10.14722/madweb.2021.23004}\\

\ifthenelse{\equal{\ADDCONTRIB}{True}}%
    {\includepdf[pages=1-12]{papers/crow.pdf}} %
    {} % 
    
\chapter{Multi-Variant Execution at the Edge}

\textbf{Javier Cabrera-Arteaga}, Pierre Laperdrix, Martin Monperrus, Benoit Baudry\\
\emph{Conference on Computer and Communications Security (CCS 2022), Workshop on Moving Target Defense (MTD)}\\\\
 \url{https://dl.acm.org/doi/abs/10.1145/3560828.3564007}\\\\

\ifthenelse{\equal{\ADDCONTRIB}{True}}%
    {\includepdf[pages=1-12]{papers/mewe_final.pdf}} % 
    {} %
    
  \chapter{Superoptimization of WebAssembly Bytecode}

  \textbf{Javier Cabrera-Arteaga}, Shrinish Donde, Jian Gu, Orestis Floros, Lucas Satabin, Benoit Baudry, Martin Monperrus\\
  \emph{Conference Companion of the 4th International Conference on Art, Science, and Engineering of Programming (Programming 2021), MoreVMs}\\\\
  \url{https://doi.org/10.1145/3397537.3397567}\\
  
  % Add Abstract
  
  \ifthenelse{\equal{\ADDCONTRIB}{True}}%
      {\includepdf[pages=1-5]{papers/souper.pdf}} % 
      {} %
    
\chapter{Scalable Comparison of JavaScript V8 Bytecode Traces}

\textbf{Javier Cabrera-Arteaga}, Martin Monperrus, Benoit Baudry\\
\emph{11th ACM SIGPLAN International Workshop on Virtual Machines and Intermediate Languages (SPLASH 2019)}\\\\
\url{https://doi.org/10.1145/3358504.3361228}\\


\ifthenelse{\equal{\ADDCONTRIB}{True}}%
    {\includepdf[pages=1-10]{papers/STRAC.pdf}} %
    {} %
    
\printindex

\end{document}
\endinput
%%
%% End of file `kth-demo.tex'.
