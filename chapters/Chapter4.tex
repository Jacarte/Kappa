\chapter{Exploiting Software Diversification for \Wasm}
\label{method}

\msection{Offensive Diversification: Malware evasion}

\wrule{Binary rewriting tools and obfuscators} The landscape for tools that can modify, obfuscate, or enhance \Wasm binaries for various has increased. 
For instance, BREWasm\cite{BREWasm} provides a comprehensive static binary rewriting framework specifically designed for \Wasm. 
Wobfuscator\cite{wobfuscator} takes a different approach, serving as an opportunistic obfuscator for Wasm-JS browser applications. 
Madvex\cite{madvex} focuses on modifying \Wasm binaries to evade malware detection, with its approach being limited to alterations in the code section of a \Wasm binary. 
Additionally, WASMixer\cite{wasmixer} obfuscates WebAssembly binaries, by including memory access encryption, control flow flattening, and the insertion of opaque predicates.


\todo{ The malware evasion paper}

\msubsection{Objective}

Test and evade the resilience of \Wasm malware detectors mentioned in \autoref{background:wasm:ecosystems}.

\msubsection{Approach}

\todo{We use wasm-mutate}
\todo{How do we use it?}
\todo{Controlled and uncontrolled diversification.}

\msubsection{Results}



\msection{Defensive Diversification: Speculative Side-channel protection}

\todo{Go around the last paper}

\msubsection{Threat model}


- Spectre timing cache attacks.

- Rockiki paper on portable side channel in browsers.


\msubsection{Approach}

- Use of wasm-mutate

\msubsection{Results}

- Diminshing of BER