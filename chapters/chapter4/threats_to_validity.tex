\msection{Threats to validity}
\label{threats}

We discuss the threats to the validity of the two use cases presented in this chapter.
We separate the threats to validity into three main categories: internal, external, and construct validity.

\msubsection{Internal validity}


\msubsection{External validity}


\msubsection{Construct validity}


\subsection{Partial input/output validation}

% We need to talk about this because, we do this checking right noe and it is probably a reason for the low count of variants.
When \tool generates a variant, it can be executed to check the input/output equivalence.
If the variant has a \_start function, both binaries, the original and the variant can be initialized. 
If the state of the memory, the globals and the stack is the same after executing the \_start function, they are partially equivalent.
%This mechanismm is already implemented in the fuzzing campaign of wasmtime.

The \_start function is easier to execute given its signature.
It does not receive parameters.
Therefore, it can be executed directly.
Yet, since a \Wasm program might contain more than one function that could be indistinctly called with and arbitrary number of parameters, we are not able to validate the whole program.
Thus, we call the checking of the initialization of a \wasm variant, a partial validation.
