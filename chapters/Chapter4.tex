\chapter{Assessing Software Diversification for \Wasm}
\label{exploit}



\chapterprecishere{\input{quotes/dk.tex}\par\raggedleft--- {\small\textup{Donald Knuth}}}

\vspace{12mm}

\lettrine[lines=3]{I}{n} this chapter, we illustrate the application of Software Diversification for both offensive and defensive purposes.
We discuss two selected use cases that demonstrate the practical applications of our contributions.
Additionally, we discuss the challenges and benefits arising from the application of Software Diversification to \Wasm.





\msection{Offensive Diversification: Malware evasion}
\label{offensive_app}

The primary malicious use of WebAssembly in browsers is cryptojacking \cite{musch2019new}. 
This is due to the essence of cryptojacking, the faster the mining, the better. 
Let us illustrate how a malicious \Wasm binary is involved into browser cryptojacking.
\autoref{fig:attack_crypto} illustrates a browser attack scenario:
a practical WebAssembly cryptojacking attack consists of three components: a WebAssembly binary, a JavaScript wrapper, and a backend cryptominer pool. 
The WebAssembly binary is responsible for executing the hash calculations, which consume significant computational resources. 
The JavaScript wrapper facilitates the communication between the WebAssembly binary and the cryptominer pool.

\begin{figure}[h]
    \centering
    \includegraphics[width=0.6\linewidth]{figures/attack_crypto.pdf}
    \caption{A remote mining pool server, a JavaScript wrapper and the \Wasm binary form the triad of a cryptojacking attack in browser clients.}
    \label{fig:attack_crypto}
\end{figure}

The aforementioned components require the following steps to succeed in cryptomining.
First, the victim visits a web page infected with the cryptojacking code. 
The web page establishes a channel to the cryptominer pool, which then assigns a hashing job to the infected browser. 
The WebAssembly cryptominer calculates thousands of hashes inside the browser. 
Once the malware server receives acceptable hashes, it is rewarded with cryptocurrencies for the mining. 
Then, the server assigns a new job, and the mining process starts over.

Both antivirus software and browsers have implemented measures to detect cryptojacking. For instance, Firefox employs deny lists to detect cryptomining activities \cite{firefoxcrypto}. 
The academic community has also contributed to the body of work on detecting or preventing WebAssembly-based cryptojacking, as outlined in \autoref{background:wasm:analysis}. 
However, malicious actors can employ evasion techniques to circumvent these detection mechanisms. 
Bhansali \etal are among the first who have investigated how WebAssembly cryptojacking could potentially evade detection \cite{10.1145/3507657.3528560}, highlighting the critical importance of this use case. 
The case illustrated in the subsequent sections uses Offensive Software Diversification for evading malware detection in \Wasm. 

\msubsection{Cryptojacking defense evasion}
\label{threat_model}


Considering the previous scenario, several techniques can be directly implemented in browsers to thwart cryptojacking by identifying the malicious \Wasm components. 
Such defense scenario is illustrated in \autoref{fig:threat_model}, where the \Wasm malicious binary is blocked in \step{3}.
The primary aim of our use case is to investigate the effectiveness of code diversification as a means to circumvent cryptojacking defenses. 
Specifically, we assess whether the following evasion workflow can successfully bypass existing security measures:

\begin{figure}
    \centering
    \includegraphics[width=0.8\linewidth]{figures/threat_model.pdf}
    \caption{Cryptojacking scenario in which the malware detection mechanism is bypassed by using an evasion technique.}
    \label{fig:threat_model}
\end{figure}


\begin{enumerate}
    
    \item The user loads a webpage infected with cryptojacking malware, which leverages network resources for execution—corresponding to \step{1} and \step{2} in \autoref{fig:threat_model}. 
    
    \item A malware detection mechanism (malware oracle) identifies and blocks malicious WebAssembly binaries at \step{3}. 
    For example, a network proxy could intercept and forward these resources to an external detection service via its API.
    
    \item Anticipating that a specific malware detection system is consistently used for defense, the attacker swiftly generates a variant of the WebAssembly cryptojacking malware designed to evade detection at \step{4}.
    
    \item The attacker delivers the modified binary instead of the original one \step{5}, which initiates the cryptojacking process and compromises the browser \step{6}. The detection method is not capable of detecting the malicious nature of the binary, and the attack is successful.
    
\end{enumerate}


\msubsection{Methodology}

Our aim is to empirically validate the workflow in \autoref{fig:threat_model}, i.e., using Offensive Software Diversification in evading malware detection systems.
To achieve this, we employ WASM-MUTATE for generating \Wasm malware variants.
In this study, we categorize malware detection mechanisms as malware oracles, which can be of two types: binary and numeric. 
A binary oracle provides a binary decision, labeling a \Wasm binary as either malicious or benign. 
In contrast, a numeric oracle returns a numerical value representing the confidence level of the detection.

\begin{definition}{Malware oracle:}
    \label{malware_oracle_def}
    A malware oracle is a detection mechanism that returns either a binary decision or a numerical value indicating the confidence level of the detection.
\end{definition}


We employ VirusTotal as a numeric oracle and MINOS \cite{MINOS} as a binary oracle. 
VirusTotal is an online service that analyzes files and returns a confidence score in the form of the number of antivirus that flag the input file as malware, thus qualifying as a numeric oracle. 
MINOS, on the other hand, converts \Wasm binaries into grayscale images and employs a convolutional neural network for classification. 
It returns a binary decision, making it a binary oracle.


We use the wasmbench dataset \cite{Hilbig2021AnES} to establish a ground truth. 
After running the wasmbench dataset through VirusTotal and MINOS, we identify 33 binaries that are: 1) flagged as malicious by at least one VirusTotal vendor and, 2) are also detected by MINOS.
Then, to simulate the evasion scenario in \autoref{fig:threat_model}, we use WASM-MUTATE to generate \Wasm binary variants to evade malware detection (\step{4} in \autoref{fig:threat_model}).
We use WASM-MUTATE in two configurations: feedback-guided and stochastic diversification.

\begin{definition}{Feedback-guided Diversification:}
    \label{controlled_def}
    In feedback-guided diversification, the transformation process of a \Wasm program is guided by a numeric oracle, which influences the probability of each transformation. For instance, WASM-MUTATE can be configured to apply transformations that minimize the oracle's confidence score. Note that feedback-guided diversification needs a numeric oracle.
\end{definition}


\begin{definition}{Stochastic Diversification:}
    \label{uncontrolled_def}
    Unlike feedback-guided diversification, in stochastic diversification, each transformation has an equal likelihood of being applied to the input \Wasm binary.
\end{definition}


Based on the two types of malware oracles and diversification configurations, we examine three scenarios:
1) VirusTotal with a feedback-guided diversification, 2) VirusTotal with an stochastic diversification, and 3) MINOS with a stochastic diversification.
Notice that, the fourth scenario with MINOS and a feedback-guided diversification is not feasible, as MINOS is a binary oracle and cannot provide the numerical values required for feedback-guided diversification.

Our evaluation focuses on two key metrics: the success rate of evading detection mechanisms in VirusTotal and MINOS across the 33 flagged binaries, and the correctness of the generated variants.

\begin{definition}{Evasion rate:} This measures the efficacy of WASM-MUTATE in bypassing malware detection systems. 
    For each flagged binary, we input it into WASM-MUTATE, configured with the selected oracle and diversification strategy. 
    We then iteratively apply transformations to the output from the preceding step. 
    This iterative process is halted either when the binary is no longer flagged by the oracle or when a maximum of 1000 stacked transformations have been applied (see \autoref{stack_transform}).
    This process is repeated with 10 random seeds per binary to simulate 10 different evasion experiments per binary.
\end{definition}

\begin{definition}{Correctness:} This verifies the functional equivalence of the variants generated by WASM-MUTATE compared to the original binary. 
    We execute the variants that entirely evade VirusTotal, using controlled and stochastic diversification configurations with WASM-MUTATE for both metrics. 
    Our selection is limited to variants that allow us to fully reproduce the three components displayed in \autoref{fig:attack_crypto}. 
    We then gather the hashes generated by the cryptojacking binaries and their generation speed, comparing these hashes with those from the original binary. 
    If the hashes match, and the variant executes without error, with the minerpool component validating the hash, we can consider the variant as functionally equivalent.
\end{definition}

\msubsection{Results}

In \autoref{offensive:results:fast}, we present a comprehensive summary of the evasion experiments presented in \cite{EVASION}, focusing on two oracles: VirusTotal and MINOS\cite{MINOS}. 
The table is organized into two main categories to separate the results for each malware oracle. 
For VirusTotal, we further subdivide the results based on the two diversification configurations we employ: stochastic and feedback-guided diversification. 
In these subsections, the columns indicate the number of VirusTotal vendors that flag the original binary as malware (\#D), the maximum number of successfully evaded detectors (Max. \#evaded), and the average number of transformations required (Mean \#trans.) for each sample. 
We highlight in bold text the values for which the stochastic diversification or feedback-guided diversification setups best, the lower, the better.
The MINOS section solely includes a column that specifies the number of transformations needed for complete evasion. 
The table has 33 + 1 rows, each representing a unique \Wasm malware study subject. 
The final row offers the median number of transformations required for evasion across our evaluated setups and oracles. 

\newcolumntype{t}{>{\columncolor{Gray}}r}
\begin{table}
    \footnotesize
    \centering
    \begin{tabular}{c | r | r l | r l | t }
        \hline
        & \multicolumn{5}{|c|}{VirusTotal} & MINOS\cite{MINOS} \\
        \hline
        Hash & \#D & \multicolumn{2}{c|}{Stochastic diversification} & \multicolumn{2}{c|}{Feedback-guided diversification} & \\
        \hline
        &  & Max. #evaded & Mean #trans. & Max. #evaded & Mean #trans. & Mean #trans. \\
        \hline\hlined
        47d29959 &                 31 &             \textbf{26} &     N/A & 19 & N/A  & 100  \\ 
        9d30e7f0 &                 30 &             \textbf{24}  &      N/A & 17 & N/A & 419  \\ 
        8ebf4e44 &                 26 &             \textbf{21} &     N/A  & 13 & N/A & 92 \\
        \hline
        c11d82d &                 20 &       20        &  \textbf{355} & 20 & 446 & 115 \\ 
        0d996462 &                 19 &     19    &  \textbf{401} & 19 & 697 & 24 \\ 
        a32a6f4b &                 18 &       18       &  635 & 18 & \textbf{625} & 1 \\
        
        
        fbdd1efa &                 18 &         18      &  \textbf{310} & 18 & 726 & 1 \\ 
        d2141ff2 &                  9 &          9      &  \textbf{461} & 9 & 781 & 81 \\ 
        aafff587 &                  6 &          6      &  484 & 6 & \textbf{331} & 1 \\
        
        
        046dc081 &                  6 &          6      &  404 & 6 & \textbf{159} & 33 \\ 
        643116ff &                  6 &          6      &  \textbf{144} & 6 & 436 & 47 \\ 
        15b86a25 &                  4 &          4      &  253 & 4 & \textbf{131} & 1 \\
        
        
        
        006b2fb6 &                  4 &           4     &  \textbf{282} & 4 & 380 & 1\\ 
        942be4f7 &                  4 &           4     &  200 & 4 & 200 & 29\\ 
        7c36f462 &                  4 &           4     &  236 & 4 & \textbf{221} & 85\\
        
        
        fb15929f &                  4 &            4    &  \textbf{297} & 4 & 475 & 1 \\ 
        24aae13a &                  4 &         4       &  \textbf{252 } & 4 & 401 & 980\\ 
        000415b2 &                  3 &         3       &  302 & 3 & \textbf{34} & 960 \\
        
        4cbdbbb1 &                  3 &          3      &  295 & 3 & \textbf{72} & 1\\ 
        65debcbe &                  2 &          2      &  131  & 2 & \textbf{33} & 38 \\ 
        59955b4c &                  2 &          2      &  130  & 2 & \textbf{33} & 38 \\
        
        
        89a3645c &                  2 &           2     &  431 & 2 & \textbf{107} & 108\\
        a74a7cb8 &                  2 &           2     &  124 & 2 & \textbf{33} & 38 \\
        119c53eb &                  2 &           2     &  104 & 2 & \textbf{18} & 1 \\
        
        089dd312 &                  2 &           2     &  153 & 2 & \textbf{123} & 68\\
        c1be4071 &                  2 &           2     &  130 & 2 & \textbf{33} & 38\\
        dceaf65b &                  2 &           2     &  140 & 2 & 132 & 66\\
        
        6b8c7899 &                  2 &            2    &  143 & 2 & \textbf{33} & 38 \\
        a27b45ef &                  2 &         2       &  145 & 2 & \textbf{33} & 33\\
        68ca7c0e &                  2 &         2       &  137  & 2 & \textbf{33} & 38\\
        
        f0b24409 &                  2 &         2       &  127  & 2 & \textbf{11} & 33 \\
        5bc53343 &                  2 &         2       &  118  & 2 & \textbf{33} & 33 \\
        e09c32c5 &                  1 &         1       &  \textbf{120}  & 1 & 488 & 15 \\
        \hline\hline
        Median &                         &         &      218  &   & 131 & 38
    \end{tabular}
    \caption{
        The table has two main categories for each malware oracle, corresponding to the two oracles we use: VirusTotal and MINOS. 
        For VirusTotal, divide the results based on the two diversification configurations: stochastic and feedback-guided diversification. 
        We provide columns that indicate the number of VirusTotal vendors that flag the original binary as malware (\#D), the maximum number of successfully evaded detectors (Max. \#evaded), and the average number of transformations required (Mean \#trans.) for each sample. 
        We highlight in bold text the values for which diversification setups are best, the lower, the better.
        The MINOS section includes a column that specifies the number of transformations needed for complete evasion. 
        The final row offers the median number of transformations required for evasion across our evaluated setups and oracles. 
    }
    \label{offensive:results:fast}
\end{table}

\begin{strategy}[Stochastic diversification to evade VirusTotal]
    \label{stochastic_div_vt}
    We execute a stochastic diversification with WASM-MUTATE, setting a limit of 1000 iterations for each binary. 
    In every iteration, we query VirusTotal to determine if the newly generated binary can elude detection. 
    We repeat this procedure with ten distinct seeds for each binary, replicating ten different evasion experiments. 
    As the stochastic diversification section of \autoref{offensive:results:fast} illustrates, we successfully produce variants that fully evade detection for 30 out of 33 binaries. 
    The average amount of iterations required to produce a variant that evades all detectors oscillates between 120 to 635 stacked transformations. 
    The mean number of iterations needed never exceeds 1000 stacked transformations. 
    However, three binaries remain detectable under the stochastic diversification setup. 
    In these instances, the algorithm fails to evade 5 out of 31, 6 out of 30, and 5 out of 26 detectors. 
    This shortfall can be attributed to the maximum number of iterations, 1000, that we employ in our experiments. 
    Increasing iterations further, however, seems unrealistic. 
    If certain transformations enlarge the binary size, a significantly large binary could become impractical due to bandwidth limitations. 
    In summary, stochastic diversification with WASM-MUTATE markedly reduces the detection rate by VirusTotal antivirus vendors for cryptojacking malware, achieving total evasion in 30 out of 33 (90\%) cases within the malware dataset. 
    %WASM-MUTATE proves capable of successfully evading detection systems in just a few minutes.    
\end{strategy}
    

\begin{strategy}[Feedback-guided diversification to evade VirusTotal]
    \label{guided_div_vt}
    stochastic diversification does not guide the diversification based on the number of evaded detectors, it is purely random, and has some drawbacks.
    For example, some transformations might suppress other transformations previously applied.
    We have observed that, by carefully selecting the order and type of transformations applied, it is possible to evade detection systems in fewer iterations.
    This can be appreciated in the results of the feedback-guided diversification part of \autoref{offensive:results:fast}.
    The feedback-guided diversification setup successfully generates variants that totally evade the detection for 30 out of 33 binaries, it is thus as good as the stochastic setup.
    Remarkably, for 21 binaries out of 30, feedback-guided needs only 40\% of the calls the stochastic diversification setup needs, demonstrating larger efficiency. 

    %In practice, a potential attacker may be limited by budget on the number of transformations applicable to the malware binary, e.g., the number of queries to VirusTotal. 
    %Additionally, the performance of the resulting binary benefits from this approach.

\end{strategy}
  
\begin{strategy}[Stochastic diversification to evade MINOS]
    \label{stochastic_div_minos}
    Relying exclusively on VirusTotal for detection could pose issues, particularly given the existence of specialized solutions for \Wasm, which differ from the general-purpose vendors within VirusTotal. 
    In \autoref{background:wasm:analysis} we highlight several examples of such solutions.
    Yet, for its simplicity, we extend this experiment by using MINOS\cite{MINOS}, an antivirus specifically designed for \Wasm. 
    The results of evading MINOS can be seen in the final column of \autoref{offensive:results:fast}.
    The bottom row of \autoref{offensive:results:fast} highlights that fewer iterations are required to evade MINOS than VirusTotal through WebAssembly diversification, indicating a greater ease in eluding MINOS.
    The stochastic diversification setup requires a median iteration count of 218 to evade VirusTotal. 
    In contrast, the feedback-guided diversification setup necessitates only 131 iterations. 
    Remarkably, a mere 38 iterations are needed for MINOS. 
    WASM-MUTATE evaded detection for 8 out of 33 binaries in a single iteration. 
    This result implies a vulnerability in the MINOS model to binary diversification.
\end{strategy}
    
%\msubsection{Efficiency and correctness results}
\vspace{5mm}
\begin{strategy}[\Wasm variants correctness]
    \label{evasion_impact}
    To evaluate the correctness of the malware variants created with WASM-MUTATE, we focused on six binaries that we could build and execute end-to-end, as these had all three components outlined in \autoref{fig:attack_crypto}. 
    We select only six binaries because the process of building and executing the binaries involves three components: the \Wasm binary, its JavaScript complement, and the miner pool. 
    These components were not found for the remaining 24 evaded binaries in the study subjects.
    For the six binaries, we then replace the original WebAssembly code with variants generated using VirusTotal as the malware oracle and WASM-MUTATE for both controlled and stochastic diversification configurations. 
    We then execute both the original and the generated variants. 
    We assess the correctness of the variants by examining the hashes they generate.
    Our findings show that all variants generated with WASM-MUTATE are correct, i.e., they generate the correct hashes and execute without error.
    Additionally, we found that 19\% of the generated variants surpassed the original cryptojacking binaries in performance.
\end{strategy}

% what's missing is to take a step back: what do those results validate? Wasm-mutate? the evasion process? both? what does this mean wrt to diversification in general?

%\todo{Do those results hold only for Spectre? do they generalize to other attacks? do you recommend binary diversification for any Wasm build pipeline as an hardening technique? how far are we from that?}

\begin{tcolorbox}[title=Reflection,boxrule=1pt,arc=.2em,boxsep=1.0mm]
    Our experiments conclusively demonstrate that WASM-MUTATE can effectively circumvent malware detection systems. 
    A possible key factor behind this is a misguided perception of resilience. 
    Malware detection is a well-known difficult problem \cite{cohen1987computer}. 
    Yet, prior research on static WebAssembly malware detection has shown an erroneous presumption: the existing of only metadata (\Wasm custom sections) obfuscation, or the complete absence of obfuscation techniques for \Wasm \cite{Minesweeper, MinerRay, SEISMIC, RAPID, MINOS}. 
    As explored in \autoref{sota:sw}, a software diversification engine can potentially function as an obfuscator. %%ORIGINAL
    The discussed use case partially demonstrates the assumption of non-existing obfuscators might be incorrect. 
    Consequently, our software diversification tools provide a viable solution for enhancing the accuracy of \Wasm malware detection systems.

    %Ultimately, our findings suggest that feedback-guided diversification is more efficient than stochastic diversification, requiring fewer transformations to avoid detection.    
\end{tcolorbox}


\begin{tcolorbox}[title=Contribution paper,boxrule=1pt,arc=.2em,boxsep=1.0mm]
    WASM-MUTATE generates correct and performant variants of WebAssembly cryptojacking that successfully evade malware detection.
    The case discussed in this section is fully detailed in Cabrera-Arteaga \etal "WebAssembly Diversification for Malware Evasion"
    \emph{at Computers \& Security, 2023}
    \url{https://www.sciencedirect.com/science/article/pii/S0167404823002067}. 
\end{tcolorbox}


% Efficiency
%We have found that 
%This improvement is attributed to WASM-MUTATE's ability to introduce code optimizations. 
%Additionally, debloating transformations, which eliminate unnecessary structures and dead code, resulted in a higher hash generation rate during the initial seconds of mining, likely due to faster compilation times. 
%This suggests that focused optimization serves as a valuable tool for evasion in browsers.
% The contrary case.
%On the contrary, 80\% of the generated variants are less efficient than the original binary, with the least efficient variant operating at only 20\% of the original hash generation rate. 
%This performance drop is primarily due to non-optimal transformations introduced by WASM-MUTATE. 
%Variants generated through stochastic diversification are generally slower.
%In summary, feedback-guided diversification yielded variants that evaded VirusTotal detection with minimal performance overhead—the worst-performing variant was only 1.93 times slower than the original.


\msection{Defensive Diversification: Speculative Side-channel protection}

As discussed in \autoref{background:wasm:ecosystems}, \Wasm is quickly becoming a cornerstone technology in backend systems. 
Leading companies like Cloudflare and Fastly are championing the integration of \Wasm into their edge computing platforms, thereby enabling developers to deploy applications that are both modular and securely sandboxed. 
These client-side \Wasm applications are generally architected as isolated, single-responsibility services, a model referred to as Function-as-a-Service (FaaS) \cite{pMendkiServerless, 1244493Jacobsson}. 
The operational flow of \Wasm binaries in FaaS platforms is illustrated in \autoref{fig:edge_model}.

\begin{figure}[h]
    \centering
    \includegraphics[width=0.8\linewidth]{figures/edge.pdf}
    \caption{\Wasm binaries on FaaS platforms. Developers can submit any \Wasm binary to the platform to be executed as a service in a sandboxed and isolated manner. Yet, \Wasm binaries are not immune to Spectre attacks.}
    \label{fig:edge_model}
\end{figure}


The fundamental advantage of using \Wasm in FaaS platforms lies in its ability to encapsulate thousands of client \Wasm binaries within a singular host process.
A developer could compile its source code into a \Wasm program suitable for the cloud platform and then submit it (\step{1} in \autoref{fig:edge_model}).
This host process is then disseminated across a network of servers and data centers (\step{2} in \autoref{fig:edge_model}). 
These platforms convert \Wasm programs into native code, which is subsequently executed in a sandboxed environment. 
Host processes can then instantiate new \Wasm sandboxes for each client function, executing them in response to specific user requests with nanosecond-level latency (\step{3} in \autoref{fig:edge_model}). 
This architecture inherently isolates \Wasm binary executions from each other as well as from the host process, enhancing security.

However, while \Wasm is engineered with a strong on security and isolation, it is not entirely immune to vulnerabilities such as Spectre attacks \cite{Spectre,Narayan2021Swivel} (\step{4} in \autoref{fig:edge_model}). 
In the sections that follow, we explore how software diversification techniques can be employed to fortify \Wasm binaries against such attacks. Dale ven

For an in-depth discussion on this topic, we direct the reader to our contribution \cite{wasmmutate}.

\msubsection{Threat model: speculative side-channel attacks}

To illustrate the threat model concerning \Wasm programs in FaaS platforms, consider the following scenarios. 
Developers, including potentially malicious actors, have the ability to submit any \Wasm binary to the FaaS platform. 
A malicious actor could then upload a \Wasm binary that, once compiled to native code, employs Spectre attacks to either leak sensitive information from the host process or violate Control Flow Integrity (CFI).
Furthermore, even if a submitted \Wasm binary is not intentionally malicious, it may still be vulnerable to Spectre attacks. 
For instance, a malicious actor could exploit this vulnerability by executing the susceptible binary through the FaaS service. 

Spectre attacks exploit hardware-based prediction mechanisms to trigger mispredictions, leading to the speculative execution of specific instruction sequences that are not part of the original, sequential execution flow. 
By taking advantage of this speculative execution, an attacker can potentially access sensitive information stored in the memory allocated to other \Wasm instance(including itself) or even the host process itself. 
This poses a significant risk, compromising both the security and integrity of the overall system.

Narayan and colleagues \cite{Narayan2021Swivel} have categorized potential Spectre attacks on \wasm binaries into three distinct types, each corresponding to a specific hardware predictor being exploited and a particular FaaS scenario: Branch Target Buffer Attacks,  Return Stack Buffer Attacks, and Pattern History Table Attacks defined as follows:

\begin{enumerate}
    \item The Spectre Branch Target Buffer (btb) attack exploits the branch target buffer by predicting the target of an indirect jump, thereby rerouting speculative control flow to an arbitrary target.
    \item  The Spectre Return Stack Buffer (rsb) attack exploits the return stack buffer that stores the locations of recently executed call instructions to predict the target of \texttt{ret} instructions.
    \item The Spectre Pattern History Table (pht) takes advantage of the pattern history table to anticipate the direction of a conditional branch during the ongoing evaluation of a condition.
\end{enumerate}


%\lipsum[1]

%\lipsum[1]

\msubsection{Methodology}

Our goal is to empirically validate that Software Diversification can effectively mitigate the risks associated with Spectre attacks in \Wasm binaries. 
The green-highlighted section in \autoref{fig:defense_model} illustrates how Software Diversification can be integrated into the FaaS platform workflow. 
The core idea is to generate unique and diverse \Wasm variants that can be randomized at the time of deployment. 
For this use case, we employ WASM-MUTATE as our tool for Software Diversification.

To empirically demonstrate that Software Diversification can indeed mitigate Spectre vulnerabilities, we reuse the \Wasm attack scenarios proposed by Narayan and colleagues in their work on Swivel \cite{Swivel}. 
Swivel is a compiler-based strategy designed to counteract Spectre attacks on \Wasm binaries by linearizing their control flow during machine code compilation. 
Our approach differs from theirs in that it is binary-based, compiler-agnostic, and platform-agnostic; we do not propose altering the deployment or toolchain of FaaS platforms. 
Although our experiments are conducted prior to submitting the \Wasm binary to the FaaS platform, we argue that \Wasm binary diversification could be implemented at any stage of the FaaS workflow.
The same argument holds by using any other diversification tool presented in this dissertation (see \autoref{tech}).


\begin{figure}[h]
    \centering
    \includegraphics[width=0.75\linewidth]{figures/edge_protected.pdf}
    \caption{Diversifying \Wasm binaries to mitigate Spectre attacks in FaaS platforms.}
    \label{fig:defense_model}
\end{figure}

\begin{table}
    \centering
    \begin{tabular}{l | l  }
        \hline
         Program &  Attack  \\
        \hline \hline
        btb\_breakout & Spectre branch target buffer (btb)  \\
        \hline
         btb\_leakage & Spectre branch target buffer(btb)  \\
        \hline
         ret2spec &  Spectre Return Stack Buffer (rsb)  \\
        \hline
        pht &  Spectre Pattern History Table (pht)  \\

%\end{adjustbox}
    \end{tabular}
    \caption{\Wasm program name and its respective attack.}
    \label{programs}
\end{table}

To measure the efficacy of WASM-MUTATE in mitigating Spectre, we diversify four \Wasm binaries proposed in the Swivel study. 
The names of these programs and the specific attacks we examine are available in \cite{programs}. 
For each of these four binaries, we generate up to 1000 random stacked transformations (see \autoref{stack_transform}) using 100 distinct seeds, resulting in a total of 100,000 variants for each original binary. 
At every 100th stacked transformation for each binary and seed, we assess the impact of diversification on the Spectre attacks by measuring the attack bandwidth for data exfiltration. 
This metric not only captures the success or failure of the attacks but also quantifies the extent to which data exfiltration is hindered. 
For example, a variant that still leaks data but does so at an impractically slow rate would be considered hardened against the attack.

\begin{definition}{Attack bandwidth:}\label{metric:ber}
    Given data $D=\{b_0, b1, ..., b_C\}$ being exfiltrated in time $T$ and $K = {k_1, k_2, ..., k_N}$ the collection of correct data bytes, the bandwidth metric is defined as:
    $$
        \frac{|b_i\text{ such that } b_i \in K|}{T}
    $$
\end{definition}


\msubsection{Results}


\autoref{attacks:impact:1} offers a graphical representation of \tool's influence on the Swivel original programs: btb\_breakout and btb\_leakage with the btb attack. 
The Y-axis represents the exfiltration bandwidth (see \autoref{metric:ber}). 
The bandwidth of the original binary under attack is marked as a blue dashed horizontal line.
In each plot, the variants are grouped in clusters of 100 stacked transformations. 
These are indicated by the green violinplots.

\begin{figure}[h]
    \centering
    \includegraphics[width=\linewidth]{plots/spectre/results.rq3.1.pdf}
    \caption{Impact of WASM-MUTATE over btb\_breakout and btb\_leakage binaries. The Y-axis denotes exfiltration bandwidth, with the original binary's bandwidth under attack highlighted by a blue marker and dashed line. Variants are clustered in groups of 100 stacked transformations, denoted by green violinplots. 
    Overall, for all 100000 variants generated out of each original program, 70\% have less data leakage bandwidth.
    After 200 stacked transformations, the exfiltration bandwidth drops to zero.
    }
  \label{attacks:impact:1}
\end{figure}

\wrule{Population Strength:} For the binaries btb\_breakout and btb\_leakage, \tool exhibits a high level of effectiveness, generating variants that leak less information than the original in 78\% and 70\% of instances, respectively.
For both programs, after applying 200 stacked transformations, the exfiltration bandwidth drops to zero.
This implies that \tool is capable of synthesizing variants that are entirely protected from the original attack.


\begin{tcolorbox}[title=Takeaway,boxrule=1pt,arc=.2em,boxsep=1.0mm]
    As indicated in \autoref{comp:table:tools}, generating a variant with 200 stacked transformations can be accomplished in just a matter of minutes.
    When scaled to the scope of a global FaaS platform, this means that a unique, fortified variant could be deployed for each machine and even for each fresh \Wasm spawned per user request.
\end{tcolorbox}






\begin{figure}[h]
    \centering    
    \includegraphics[width=\linewidth]{plots/spectre/results.rq3.2.pdf}
    \caption{Impact of WASM-MUTATE over ret2spec and pht binaries. The Y-axis denotes exfiltration bandwidth, with the original binary's bandwidth under attack highlighted by a blue marker and dashed line. Variants are clustered in groups of 100 stacked transformations, denoted by green violinplots. 
    Overall, for both programs approximately 70\% of the variants have less data leakage bandwidth.}
  \label{attacks:impact:2}
\end{figure}


% \todo{replace with violin plots}
As illustrated in \autoref{attacks:impact:2}, similarly to \autoref{attacks:impact:1}, WASM-MUTATE significantly impacts the programs ret2spec and pht when subjected to their respective attacks. 
In 76\% of instances for ret2spec and 71\% for pht, the generated variants demonstrated reduced attack bandwidth compared to the original binaries.
The plots reveal that a notable decrease in exfiltration bandwidth occurs after applying at least 100 stacked transformations. 
While both programs show signs of hardening through reduced attack bandwidth, this effect is not immediate and requires a substantial number of transformations to become effective. 
Additionally, the bandwidth distribution is more varied for these two programs compared to the two previous ones.
Our analysis suggests a correlation between the reduction in attack bandwidth and the complexity of the binary being diversified. 
Specifically, ret2spec and pht are substantially larger programs, containing over 300,000 instructions, compared to btb\_breakout and btb\_leakage, which have fewer than 800 instructions. 
Therefore, given that WASM-MUTATE performs incremental transformations, the probability of affecting critical components to hinder attacks decreases in larger binaries.

\wrule{Managed memory impact:} The success in diminishing exfiltration is explained by the fact that \tool synthesizes variants that effectively alter memory access patterns. 
We have identified four primary factors responsible for the divergence in memory accesses among \tool generated variants.
First, modifications to the binary layout—even those that don't affect executed code—inevitably alter memory accesses within the program's stack. 
Specifically, \tool generates variants that modify the return addresses of functions, which consequently leads to differences in execution flow and memory accesses.
Second, one of our rewriting rules incorporates artificial global values into \wasm binaries. 
Since these global variables are inherently manipulated via the stack, and given that the stack is located within linear memory, their access inevitably affects the managed memory (see \autoref{managed_unmanaged}).
Third, \tool injects 'phantom' instructions which don't aim to modify the outcome of a transformed function during execution. 
These intermediate calculations trigger the spill/reload component of the wasmtime compiler, varying spill and reload operations. 
In the context of limited physical resources, these operations temporarily store values in memory for later retrieval and use, thus creating diverse managed memory accesses(see the example at \autoref{custom}).
Finally, certain rewriting rules implemented by WASM-MUTATE replicate fragments of code, e.g., performing commutative operations. 
These code segments may contain memory accesses, and while neither the memory addresses nor their values change, the frequency of these operations does.

% More fine grained
\wrule{Disrupting accurate timers:} Cache timing side-channel attacks, including for the four binaries analyzed in this case study, depend on precise timers to measure cache access times. 
Disrupting these timers can effectively neutralize the attack. 
For example, in other contexts, Firefox employs a strategy to counter timing attacks by randomizing its built-in JavaScript timer \cite{10.1007/978-3-319-70972-7_13}. 
WASM-MUTATE inherently adopts a similar approach, introducing perturbations in the timing steps of \wasm variants in case they are malicious. 
This is illustrated in \autoref{example:timer} and \autoref{example:timer2}, where the former shows the original time measurement and the latter presents a variant with \tool-introduced operations.
WASM-MUTATE is particularly effective in disrupting cache access timers. 
By introducing additional instructions, the inherent randomness in the time measurement of a single or a few instructions is amplified, thereby reducing the timer's accuracy. 

\input{snippets/spectre/timer}

\wrule{Padding speculated instructions:} Additionally, CPUs have a limit on the number of instructions they can cache. 
WASM-MUTATE injects instructions to potentially exceed this limit, effectively disabling the speculative execution of memory accesses. 
This approach is akin to padding \cite{padding}, as demonstrated in \autoref{example:padding} and \autoref{example:padding2}.
This padding disrupts the binary code's layout in memory, hindering the attacker's ability to initiate speculative execution. 
Even if speculative execution occurs, the memory access does not proceed as the attacker intended.

\input{snippets/spectre/padding}

\wrule{Controlled vs Uncontrolled diversification:} We observed that the exfiltration bandwidth tends to increase in variants with only a few transformations. 
This suggests that not all transformations uniformly contribute to reducing data leakage. 
Several key factors contribute to this phenomenon.
First, as emphasized previously in \autoref{offensive_app}, uncontrolled diversification can be counterproductive if a specific objective, e.g., if a cost function, is not established at the beginning of the diversification process.
Second, while some transformations yield distinct \wasm binaries, their compilation produces identical machine code.
Transformations that are not preserved(see \autoref{discussion}) undermine the effectiveness of diversification.
For example, incorporating random \texttt{nop} operations directly into \wasm does not modify the final machine code as the \texttt{nop} operations are often removed by the compiler.
The same phenomenon is observed with transformations to custom sections of \Wasm binaries.
Additionally, it is important to note that transformed code doesn't always execute, i.e., \tool may generate dead code.





% \subsection{Deoptimization}

\begin{tcolorbox}[title=Contribution paper,boxrule=1pt,arc=.2em,boxsep=1.0mm]
    Software diversification crafts \Wasm binaries that are resilient to Spectre-like attacks. 
    By integrating a software diversification layer into \Wasm binaries deployed on Function-as-a-Service (FaaS) platforms, security can be significantly bolstered. 
    This approach allows for the deployment of unique and diversified \Wasm binaries, potentially utilizing a distinct variant for each cloud node, thereby enhancing the overall security.
    The case discussed in this section is fully detailed in Cabrera-Arteaga \etal "WASM-MUTATE: Fast and Effective Binary Diversification for WebAssembly"
    \emph{Under review}
    \url{https://arxiv.org/pdf/2309.07638.pdf}. 
\end{tcolorbox}




\msection{Intrinsic properties of diversification}
\label{exploit:discussion}
In \autoref{exploit:defensive}, we have noted an increasing trend of exfiltration bandwidth in certain variants. 
\autoref{offensive_app} presents a similar case, indicating that without a clear objective in the diversification process, uncontrolled diversification can be counterproductive. 
This implies that not all transformations contribute equally to the diversification objectives of \Wasm.


\wrule{Preservation:}
Some transformations yield distinct \wasm binaries, yet their JIT compilation produces identical machine code.
Non-preserved transformations undermine the effectiveness of diversification, as discussed in \autoref{discussion}.
Incorporating random \texttt{nop} operations directly into \wasm, for instance, does not alter the final machine code because JIT compilers frequently eliminate these \texttt{nop} operations.
This phenomenon is also observed with transformations to the custom sections of \Wasm binaries.
Identical machine code, even when their \wasm variants are different, can be detected by malware detectors.
On the other hand, side-channel attacks occur at the machine code level, making \wasm variants preservation essential code for successful diversification.


\wrule{Dead code addition:} 
Transformed code may not always execute. 
For example, Software Diversification might generate dead code or introduce a new function that the original program does not execute. 
This is beneficial for static analysis, whether for avoiding reverse engineering or evading static malware detection. 
However, dynamic analysis tools can identify this type of variant. 
This reduces the effectiveness of evasion. 
Furthermore, the inclusion of non-executing dead code does not affect side-channels, thereby not strengthening against attacks.
 
% More fine grained
\wrule{Disrupting accurate timers:} Cache timing side-channel attacks, including for the four binaries analyzed in \autoref{exploit:defensive}, depend on precise timers to measure cache access times. 
Disrupting these timers can effectively neutralize the attack \cite{JStimers}. 
The \wasm variants inherently adopts a similar approach, introducing perturbations in the timing steps of \wasm variants in case they are malicious. 
This is illustrated in \autoref{example:timer} and \autoref{example:timer2}, where the former shows the original time measurement and the latter presents a variant with introduced operations.
Remarkably, WASM-MUTATE is particularly effective in disrupting cache access timers. 
By introducing additional instructions, the inherent randomness in the time measurement of a single or a few instructions is amplified, thereby reducing the timer's accuracy. 

\input{snippets/spectre/timer}

\todo{Recheck term-citation padding here}

\wrule{Padding speculated instructions:} Additionally, CPUs have a limit on the number of instructions they can cache. 
Diversification injects instructions to potentially exceed this limit, effectively disabling the speculative execution of memory accesses. 
This approach is akin to padding \cite{padding}, as demonstrated in \autoref{example:padding} and \autoref{example:padding2}.
This padding disrupts the binary code's layout in memory, hindering the attacker's ability to initiate speculative execution. 
Even if speculative execution occurs, the memory access does not proceed as the attacker intended.

\input{snippets/spectre/padding}





% \msection{Threats to validity}
\label{threats}

We discuss the threats to the validity of the two use cases presented in this chapter.
We separate the threats to validity into three main categories: internal, external, and construct validity.

\msubsection{Internal validity}


\msubsection{External validity}


\msubsection{Construct validity}


\subsection{Partial input/output validation}

% We need to talk about this because, we do this checking right noe and it is probably a reason for the low count of variants.
When \tool generates a variant, it can be executed to check the input/output equivalence.
If the variant has a \_start function, both binaries, the original and the variant can be initialized. 
If the state of the memory, the globals and the stack is the same after executing the \_start function, they are partially equivalent.
%This mechanismm is already implemented in the fuzzing campaign of wasmtime.

The \_start function is easier to execute given its signature.
It does not receive parameters.
Therefore, it can be executed directly.
Yet, since a \Wasm program might contain more than one function that could be indistinctly called with and arbitrary number of parameters, we are not able to validate the whole program.
Thus, we call the checking of the initialization of a \wasm variant, a partial validation.


\section*{Conclusions}
In this chapter, we explore Offensive and Defensive Software Diversification applied to \Wasm.
Offensive Software Diversification highlights both the potential and the latent security risks in applying Software Diversification to \Wasm malware. 
Our findings suggest potential enhancements to the automatic detection of cryptojacking malware in WebAssembly, e.g., by stressing their resilience with \Wasm malware variants. 
Conversely, Defensive Software Diversification serves as a proactive guard, specifically designed to mitigate the risks associated with Spectre attacks. 

Moreover, we have conducted experiments with various use cases that are not shown in this chapter.
For instance, CROW \cite{CROW} excels in generating \Wasm variants that minimize side-channel noise, thereby bolstering defenses against potential side-channel attacks. 
Alternatively, deploying multivariants from MEWE \cite{MEWE} can thwart high-level timing-based side-channels \cite{morgan2015web}. 
Specifically, we conducted experiments on the round-trip times of the generated multivariants and concluded that, at a high level, the timing side-channel information cannot discriminate between variants. 
