
\msection{Discussion}

\msubsection{Accomanying artifacts}
%\msubsection{Implementation}

\tool is implemented in Rust, comprising approximately, 10 thousands lines of Rust code. 
We leverage the capabilities of the wasm-tools project of the bytecodealliance for parsing and transforming WebAssembly binary code. 
Specifically, we utilize the wasmparser and wasm-encoder modules for parsing and encoding Wasm binaries, respectively.
The implementation of \tool is publicly available for future research and can be found at \repourl.

\todo{Comparison of the approaches.}
\todo{Add the cool table.}

% Move the following
%Other solutions would have been to diversify at the source-code level or the \wasm\ binary level. However, these facts would limit the applicability of our work.
%Our approach is more general as diversification also will work for other LLVM backends.

% Why and main change
\todo{This is contradictory to our binary solution.}
We use the superdiversifier idea of Jacob and colleagues to implement CROW because of two main reasons.
First, the code replacements generated by this technique outperform diversification strategies based on handwritten rules . 
Concretely, we can control the quality of the generated codes. 
Besides, CROW always generates equivalent programs because it is based on a solver to check for equivalence. 
Second, there is a battle-tested superoptimizer for LLVM, Souper \cite. 
This latter makes it feasible the construction of a generic LLVM superdiversifier. 