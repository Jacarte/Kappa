
\msection{Comparing CROW, MEWE, and WASM-MUTATE}

In this section, we compare CROW, MEWE, and WASM-MUTATE, highlighting their key differences. 
These distinctions are summarized in \autoref{comp:table:tools}. 
The table is organized into columns that represent attributes of each tool: the tool's name, input format, core diversification strategy, number of variants generated within an hour, targeted sections of the \Wasm binary for diversification, strength of the generated variants, and the security applications of these variants. 
Each row in the table corresponds to a specific tool. 
The \emph{Variant strength} accounts for the capability of each tool on generating variants that are preserved after the JIT compilation of V8 and wasmtime in average.
For example, a higher value of the \emph{Variant strength} indicates that the generated variants are not reversed by JIT compilers, ensuring that the diversification is preserved in an end-to-end scenario of a \Wasm program, i.e. from the source code to its final execution.
Notice that, the data and insights presented in the table are sourced from the respective papers of each tool and, from the previous discussion in this chapter.


% \msubsection{Technology and approach}
%Based on the first two columns of \autoref{comp:table:tools},
CROW is a compiler-based strategy, needing access to the source code or its LLVM IR representation to work. 
Its core is an enumerative synthesis implementation with functionallity verification using SMT solvers, ensuring the functional equivalence of the generated variants.
In addition, MEWE extends the capabilities of CROW, utilizing the same underlying technology to create program variants. 
It goes a step further by packaging the LLVM IR variants into a Wasm multivariant, providing MVE through execution path randomization.
Both CROW and MEWE are fully automated, requiring no user intervention besides the input source code.
WASM-MUTATE, on the other hand, is a binary-based tool.
It uses a set of rewriting rules and the input \wasm binary to generate program variants, centralizing its core around random e-graph traversals. 
Remarkably, WASM-MUTATE removes the need for compiler adjustments, offering compatibility with any existing \Wasm binary. 


We have observed several interesting phenomena when aggregating the empirical data presented in the corresponding papers of CROW, MEWE and WASM-MUTATE \cite{CROW,MEWE, wasmmutate}. 
This can be appreciated in the fourth, fifth and sixth columns of \autoref{comp:table:tools}.
We have observed that WASM-MUTATE generates more unique variants in one hour than CROW and MEWE in at least one order of magnitude.
This is mainly because of three reasons.
First, CROW and MEWE rely on SMT solvers to prove functionally equivalence, placing a bottleneck when generating variants.
Second, CROW and MEWE generation capabilities are limited by the \emph{overlapping} phenomenon discussed in \autoref{section:crow:example}.
Third, WASM-MUTATE can generate variants in any part of the \wasm binary, while CROW and MEWE are limited to the code and function sections.


On the other hand, CROW and MEWE, by using enumerative synthesis, ensure that the generated variants are \temph{preserved}. 
In other words, the transformations generated out of CROW and MEWE are virtually irreversible by JIT compilers, such as V8 and wasmtime.
This phenomenon is highlighted in the \emph{Variants strength} column of \autoref{comp:table:tools}, where we show that CROW and MEWE generate variants with 96\% of preservation against 75\% of WASM-MUTATE.
High preservation is especially important where the preservation of the diversification is crucial, e.g. to hinder reverse engineering.





%% 
\definecolor{Gray}{gray}{0.95}
\newcolumntype{a}{>{\columncolor{Gray}}p{0.13\linewidth}}
\newcolumntype{b}{>{\columncolor{Gray}}p{0.22\linewidth}}
\begin{landscape}
    


    \begin{table}
        \begin{tabular}{ p{0.1\linewidth} | a | a | p{0.08\linewidth} | p{0.1\linewidth} | p{0.08\linewidth} | b} 
            \hline
            \textbf{Tool} & \textbf{Input} & \textbf{Core} & \textbf{Variants in 1h} & \textbf{Target}  & \textbf{Variants Strength} & \textbf{Security applications} \\
            \hline \hline
            CROW & Source code or LLVM Ir & Enumerative synthesis with functional equivalence proved through SMT solvers & > 1k & Code section  & \textbf{96\%} & Hinders Static analysis reverse engineering.  \\
            \hline
            MEWE & Source code or LL
            VM Ir & CROW, Multivariant execution  & > 1k & Code and Function sections  & \textbf{96\%} & Hinders, static and dynamic analysis reverse engineering and, web timing-based attacks.  \\
            \hline
            WASM-MUTATE &  \textbf{Wasm binary} & hand-made rewriting rules, e-graph random traversals & \textbf{> 10k}  & \textbf{Any Wasm section}  & 76\% & Hinders signature-based identification, and cache timing side-channel attacks. \\
            
        \end{tabular}
        \caption{Comparing CROW, MEWE and WASM-MUTATE. The table columns are: the tool's name, input format, core diversification strategy, number of variants generated within an hour, targeted sections of the \Wasm binary, strength of the generated variants, and the security applications of these variants. 
        The \emph{Variant strength} accounts for the capability of each tool on generating variants that are preserved after the JIT compilation of V8 and wasmtime in average.
        Our three technical contributions are complementary tools that can be combined.
        \label{comp:table:tools}}
    \end{table}
\end{landscape}

\begin{tcolorbox}[title=Takeaway,boxrule=1pt,arc=.2em,boxsep=1.0mm]
    
    Our three technical contributions serve as complementary tools that can be combined.
    For instance, when the source code for a \Wasm binary is either non-existent or inaccessible, WASM-MUTATE offers a viable solution for generating code variants. 
    On the other hand, CROW and MEWE excel in scenarios where high preservation is crucial.
\end{tcolorbox}


\msubsection{Security applications}


The final column of \autoref{comp:table:tools} emphasizes the security benefits derived from the variants produced by our three key technical contributions. 
One immediate advantage of altering the structure of \Wasm binaries across different variants is the mitigation of signature-based identification, thereby enhancing resistance to static reverse engineering.
Additionally, our tools generate a diverse array of code variants that are highly preserved. 
This implies that these variants, each with their unique \Wasm code, retain their distinct characteristics even after being translated into machine code by JIT compilers. 
This high level of preservation significantly mitigates the risks associated with side-channel attacks that target specific machine code instructions, such as port contention attacks \cite{10.1145/3488932.3517411}.
For instance, if a \Wasm binary is transformed in such a manner that its resulting machine code instructions differ from the original, it becomes more challenging for a side-channel attack. 
Conversely, if the compiler translates the variant into machine code that closely resembles the original, the side-channel attack could still exploit those instructions to extract information about the original \Wasm binary.

Altering the layout of a \Wasm program inherently influences its managed memory during runtime (see \autoref{managed_unmanaged}).
This phenomenon is especially important for CROW and MEWE, given that they do not directly address the \Wasm memory model.
Significantly, CROW and MEWE considerably alter the managed memory by modifying the layout of the \Wasm program.
For example, the \emph{constant inferring} transformations significantly alter the layout of program variants, affecting unmanaged memory elements such as the returning address of a function.
Furthermore, WASM-MUTATE not only affects managed memory through changes in the \Wasm program layout.
It also adds rewriting rules to transform unmanaged memory instructions.
Memory alterations, either to the unmanaged or managed memories, have substantial security implications, by eliminating potential cache timing side-channels \cite{Swivel}.


Last but not least, our technical contributions enhance security against web timing-based attacks \cite{morgan2015web} by creating variants that exhibit a wide range of execution times, including faster variants compared to the original program. 
This strategy is especially prominent in MEWE’s approach, which develops multivariants functioning on randomizing execution paths, thereby thwarting attempts at timing-based inference attacks \cite{DBLP:conf/ndss/SchnitzlerKBP23}. 
Adding another layer benefit from MEWE, the integration of diverse variants into multivariants can potentially disrupt dynamic reverse engineering tools such as symbolic executors \cite{wasmixer}. 
Concretely, different control flows through a random discriminator, exponentially increase the number of possible execution paths, making multivariant binaries virtually unexplorable.


\begin{tcolorbox}[title=Takeaway,boxrule=1pt,arc=.2em,boxsep=1.0mm]
    CROW, MEWE and WASM-MUTATE generate \Wasm variants that can be used to enhance security. 
    Overall, they generate variants that are suitable for hardening static and dynamic analysis, side-channel attacks, and, to thwart signature-based identification. 
\end{tcolorbox}


\msection{Conclusions}

In this chapter, we discuss the technical specifics underlying our primary technical contributions.
We elucidate the mechanisms through which CROW generates program variants.
Subsequently, we discuss MEWE, offering a detailed examination of its role in forging MVE for \Wasm. 
We also explore the details of WASM-MUTATE, proposing a novel e-graph traversal algorithm to fast spawn \wasm program variants. 
Remarkably, we undertake a comparative analysis of the three tools, highlighting their respective benefits and limitations, alongside the potential security applications of the generated \wasm variants. 

In \autoref{exploit}, we present two use cases that support the exploitation of these tools.
\autoref{exploit} serves to bridge theory with practice, showcasing the tangible impacts and benefits realized through the deployment of CROW, MEWE, and WASM-MUTATE.
