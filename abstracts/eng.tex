
\Wasm has become the fourth officially recognized web language, allowing web browsers to adapt native applications for Web.
Moreover, \Wasm has developed into a critical component of backend scenarios such as edge computing and cloud computing.
Nowadays, \Wasm is used in a wide range of applications, including web browsers, blockchain, and cloud computing.
While security was a primary focus in its design, \Wasm remains vulnerable to attacks, including side-channels and memory corruption.
In addition, \Wasm has been exploited to transport malware, especially in instances of browser cryptojacking.
Remarkably, the predictability of the \Wasm ecosystem, including its users and the programs it hosts, is exceedingly high.
This predictability can exacerbate the impact of a vulnerability within these ecosystems.
For example, a flaw in a web browser, instigated by a faulty \Wasm program, could potentially affect millions of users.


This thesis aims to enhance the security of the \Wasm ecosystem through the introduction of methods and tools for Software Diversification.
Software Diversification is a strategy designed to augment the cost of exploitation by rendering the software less predictable.
By automatically generating numerous variants of a program, we can decrease predictability within ecosystems.
These variants harden observable properties typically utilized to carry out attacks.
For instance, we can generate variants of a program with diverse memory layouts and control-flow graphs, thereby strengthening code analysis, execution traces, and execution times.
Yet, in the context of \Wasm, Software Diversification has not been explored.



We present three pioneering tools to the community: CROW, MEWE, and WASM-MUTATE. 
Each tool is specifically designed to address a unique aspect of Software Diversification. 
Moreover, these tools synergistically enhance each other. 
We furnish empirical evidence that Software Diversification is applicable to \Wasm programs in two distinct manners: Offensive and Defensive Software Diversification. 
Our investigation into Offensive Software Diversification in \Wasm reveals potential avenues for improving the detection of \Wasm malware. 
In contrast, our experiments in Defensive Software Diversification demonstrate that \Wasm programs can be strengthened against side-channel attacks, specifically against the Spectre attack.


