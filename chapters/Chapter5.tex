
\chapter{Conclusions and Future Work}
\label{results}

%\lipsum[1]

\section{Summary of technical contributions}

%\lipsum[1]

%\lipsum[2]

%\lipsum[3]

\section{Summary of empirical findings}

%\lipsum[1]

%\lipsum[1]

%\lipsum[1]

\section{Future Work}

%\lipsum[1]

Moreover, the \Wasm ecosystem is still in its infancy compared to more mature programming environments. 
A 2021 study by Hilbig et al. found only 8,000 unique \Wasm binaries globally\cite{Hilbig2021AnES}, a fraction of the 1.5 million and 1.7 million packages available in npm and PyPI, respectively. 
This limited dataset poses challenges for machine learning-based analysis tools, which require extensive data for effective training. 
The scarcity of \Wasm programs also exacerbates the problem of software monoculture, increasing the risk of compromised \Wasm programs being consumed\cite{usenixWasm2020}. 
This dissertation aims to mitigate these issues by introducing a comprehensive suite of tools designed to enhance \Wasm security through Software Diversification and to improve testing rigor within the ecosystem.

%\lipsum[1]