
\msection{Comparing CROW, MEWE, and WASM-MUTATE}

In this section, we compare CROW, MEWE, and WASM-MUTATE, highlighting their key differences. 
These distinctions are summarized in \autoref{comp:table:tools}. 
The table is organized into columns that represent attributes of each tool: the tool's name, input format, core diversification strategy, number of variants generated within an hour, targeted sections of the \Wasm binary for diversification, strength of the generated variants, and the security applications of these variants. 
Each row in the table corresponds to a specific tool. 
Notice that, the data and insights presented in the table are sourced from the respective papers of each tool and, from the previous discussion in this chapter.


\begin{landscape}
    
    \begin{table}
        \begin{tabular}{p{0.08\linewidth} | p{0.1\linewidth} | p{0.1\linewidth
            } | p{0.06\linewidth} | p{0.1\linewidth} | p{0.07\linewidth} | p{0.25\linewidth} } 
            Tool & Input & Core & Variants in 1h & Target  & Variants Strength & Security applications \\
            \hline \hline
            CROW & Source code or LLVM Ir & Enumerative synth. & > 1k & Code section  & \textbf{96\%} & Resilience against: signature-based identification, static analysis and side-channel attacks.  \\
            \hline
            MEWE & Source code or LL
            VM Ir & CROW + Execution path randomization  & > 1k & Code + Function sections  & \textbf{96\%} & Resilience against: signature-based identification, static and dynamic analysis, web timing-based attacks.  \\
            \hline
            WASM-MUTATE & rewriting rules + \textbf{Wasm bin.} & e-graph random traversals & \textbf{> 10k}  & \textbf{Any Wasm part}  & 76\% & Resilience against: signature-based identification, static analysis, fingerprinting and timing side-channel attacks. \\
            
        \end{tabular}
        \caption{Comparing CROW, MEWE and WASM-MUTATE. The table columns are: the tool's name, input format, core diversification strategy, number of variants generated within an hour, targeted sections of the \Wasm binary, strength of the generated variants, and the security applications of these variants. 
        The \emph{Variant strength} accounts for the capability of each tool on generating variants that are preserved after the JIT compilation of V8 and wasmtime in average.
        Our three technical contributions are complementary tools that can be combined.
        \label{comp:table:tools}}
    \end{table}
\end{landscape}

% \msubsection{Technology and approach}
%Based on the first two columns of \autoref{comp:table:tools},
CROW is a compiler-based strategy, needing access to the source code or its LLVM IR representation to work. 
Its core is an enumerative synthesis implementation with functionallity verification using SMT solvers, ensuring the functional equivalence of the generated variants.
%This approach lays the groundwork for a universal LLVM superdiversifier, potentially extending its applications and adaptability to other technologies.
In addition, MEWE extends the capabilities of CROW, utilizing the same underlying technology to create program variants. 
It goes a step further by packaging the LLVM IR variants into a Wasm multivariant, providing MVE through execution path randomization.
Both CROW and MEWE are fully automated, requiring no user intervention besides the input source code.
WASM-MUTATE, on the other hand, is a semi-automated, binary-based tool.
It needs a set of rewriting rules and the \wasm binary as inputs to generate program variants, centralizing its core around random e-graph traversals. 
%This approach facilitates the creation of a pool of \Wasm program variants through the meticulous application of rewriting rules on an e-graph data structure. 
Remarkably, WASM-MUTATE removes the need for compiler adjustments, offering compatibility with any existing \Wasm binary. 
%Moreover, it highlights how extending intermediate representations could establish a general framework for binary rewriting in \Wasm.


%\msubsection{Strength of the generated variants}

%We analyze our tools through the lens of the following properties: the number of variants generated in one hour, the part of the \wasm binary diversified, the overhead introduced by the diversification and the strength of the generated variants as the \emph{Strength of the generated variants}.
We draw several interesting phenomena when aggregating the data presented in the corresponding papers of CROW, MEWE and WASM-MUTATE \cite{CROW,MEWE, wasmmutate}. 
This can be appreciated in the fourth, fifth and sixth columns of \autoref{comp:table:tools}.
We have observed that WASM-MUTATE generates more unique variants in one hour than CROW and MEWE in at least one order of magnitude.
This is mainly because WASM-MUTATE can generate variants in any part of the \wasm binary, while CROW and MEWE are limited to the code and function sections.
In addition, CROW and MEWE generation capabilities are limited by the \emph{overlapping} phenomenon discussed in \autoref{section:crow:example}.
On the other hand, CROW and MEWE, by using enumerative synthesis, can exceed handcrafted optimizations \cite{Sasnauskas2017Souper:Superoptimizer}, ensuring that the generated variants are \temph{preserved}. 
In other words, the transformations generated out of CROW and MEWE are virtually irreversible by JIT compilers, such as V8 and wasmtime.
This phenomenon is highlighted in the \emph{Variants strength} column of \autoref{comp:table:tools}, where we show that CROW and MEWE generate variants with 96\% of preservation against 75\% of WASM-MUTATE.

\msubsection{Security applications}

The last column of \autoref{comp:table:tools} highlights the security applications of the variants generated by our three technical contributions.
Our tools generate many different and highly preserved code variants. 
This means that these variants, each with unique \Wasm codes, maintain their distinctiveness even after JIT compilers translate them into machine codes (see \autoref{background:wasm:ecosystems}). 
The preservation feature significantly reduces the impact of side-channel attacks that exploit specific machine code instructions, e.g., port contention \cite{10.1145/3488932.3517411}.
Besides, the preserved transformations of the generated variants serve to reduce potential attack surfaces, thereby impeding signature-based identification.

Altering the layout of a \Wasm program inherently influences its managed memory during runtime, a component not overseen by the \Wasm program itself (see \autoref{managed_unmanaged}).
This pehomenon is especially important for CROW and MEWE, given that they do not directly address the \Wasm memory model.
Significantly, CROW and MEWE considerably alter the managed memory by modifying the layout of the \Wasm program.
For example, the \emph{constant inferring} transformations significantly alter the layout of program variants, affecting unmanaged memory elements such as the returning address of a function.
Furthermore, WASM-MUTATE not only affects managed memory through changes in the \Wasm program layout.
It also adds rewriting rules to transform unmanaged memory instructions.
Memory alterations, either to the unmanaged or managed memories, have substantial security implications, by eliminating potential jump points that facilitate malicious activities within the binary \cite{Swivel}.


Besides, our technical contributions enhance security against timing-based attacks by creating variants that exhibit a wide range of execution times. 
This strategy is especially prominent in MEWE’s approach, which develops multivariants functioning on randomizing execution paths, thereby thwarting attempts at timing-based inference attacks \cite{DBLP:conf/ndss/SchnitzlerKBP23}. 
Adding another layer benefit, the integration of diverse variants into multivariants can potentially disrupt dynamic analysis tools such as symbolic executors \cite{wasmixer}. 
Concretely, different control flows through a random discriminator, exponentially increase the number of possible execution paths, making multivariant binaries virtually unexplorable.


\begin{tcolorbox}[title=Key Takeaway,boxrule=1pt,arc=.2em,boxsep=1.0mm]
    
    Our three technical contributions serve as complementary tools that can be combined to create a more comprehensive and robust software diversification strategy. 
    For instance, when the source code for a \Wasm binary is either non-existent or inaccessible, WASM-MUTATE offers a viable solution for generating code variants. 
    On the other hand, CROW and MEWE excel in scenarios where high preservation is crucial, particularly when the generated variants may be subject to further analysis. Furthermore, WASM-MUTATE can benefit from the enumerative synthesis techniques employed by CROW and MEWE. 
    Specifically, WASM-MUTATE could incorporate the transformations generated by these tools as rewriting rules.
\end{tcolorbox}


\msection{Conclusions}

In this chapter, we discuss the technical specifics underlying our primary technical contributions.
We elucidate the mechanisms through which CROW generates program variants.
Subsequently, we discuss MEWE, offering a detailed examination of its role in forging MVE for \Wasm. 
We also explore the details of WASM-MUTATE, highlighting its pioneering utilization of an e-graph traversal algorithm to spawn \wasm program variants. 
Remarkably, we undertake a comparative analysis of the three tools, highlighting their respective benefits and limitations, alongside the potential security applications of the generated \wasm variants. 

In \autoref{chapter:method}, we present four use cases that support the exploitation of these tools.
\autoref{chapter:method} serves to bridge theory with practice, showcasing the tangible impacts and benefits realized through the deployment of CROW, MEWE, and WASM-MUTATE.
