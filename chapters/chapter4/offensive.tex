\msection{Offensive Diversification: Malware evasion}
\label{offensive_app}

The primary malicious use of WebAssembly in browsers is cryptojacking \cite{musch2019new}. 
This is due to the essence of cryptojacking, the faster the mining, the better. 
Let us illustrate how a malicious \Wasm binary is involved into browser cryptojacking.
\autoref{fig:attack_crypto} illustrates a browser attack scenario:
a practical WebAssembly cryptojacking attack consists of three components: a WebAssembly binary, a JavaScript wrapper, and a backend cryptominer pool. 
The WebAssembly binary is responsible for executing the hash calculations, which consume significant computational resources. 
The JavaScript wrapper facilitates the communication between the WebAssembly binary and the cryptominer pool.

\begin{figure}[h]
    \centering
    \includegraphics[width=0.6\linewidth]{figures/attack_crypto.pdf}
    \caption{A remote mining pool server, a JavaScript wrapper and the \Wasm binary form the triad of a cryptojacking attack in browser clients.}
    \label{fig:attack_crypto}
\end{figure}

The aforementioned components require the following steps to succeed in cryptomining.
First, the victim visits a web page infected with the cryptojacking code. 
The web page establishes a channel to the cryptominer pool, which then assigns a hashing job to the infected browser. 
The WebAssembly cryptominer calculates thousands of hashes inside the browser. 
Once the malware server receives acceptable hashes, it is rewarded with cryptocurrencies for the mining. 
Then, the server assigns a new job, and the mining process starts over.

Both antivirus software and browsers have implemented measures to detect cryptojacking. For instance, Firefox employs deny lists to detect cryptomining activities \cite{firefoxcrypto}. 
The academic community has also contributed to the body of work on detecting or preventing WebAssembly-based cryptojacking, as outlined in \autoref{background:wasm:analysis}. 
However, malicious actors can employ evasion techniques to circumvent these detection mechanisms. 
Bhansali \etal are among the first who have investigated how WebAssembly cryptojacking could potentially evade detection \cite{10.1145/3507657.3528560}, highlighting the critical importance of this use case. 
The case illustrated in the subsequent sections uses Offensive Software Diversification for evading malware detection in \Wasm. 

\msubsection{Cryptojacking defense evasion}
\label{threat_model}


Considering the previous scenario, several techniques can be directly implemented in browsers to thwart cryptojacking by identifying the malicious \Wasm components. 
Such defense scenario is illustrated in \autoref{fig:threat_model}, where the \Wasm malicious binary is blocked in \step{3}.
The primary aim of our use case is to investigate the effectiveness of code diversification as a means to circumvent cryptojacking defenses. 
Specifically, we assess whether the following evasion workflow can successfully bypass existing security measures:

\begin{figure}
    \centering
    \includegraphics[width=0.8\linewidth]{figures/threat_model.pdf}
    \caption{Cryptojacking scenario in which the malware detection mechanism is bypassed by using an evasion technique.}
    \label{fig:threat_model}
\end{figure}


\begin{enumerate}
    
    \item The user loads a webpage infected with cryptojacking malware, which leverages network resources for execution—corresponding to \step{1} and \step{2} in \autoref{fig:threat_model}. 
    
    \item A malware detection mechanism (malware oracle) identifies and blocks malicious WebAssembly binaries at \step{3}. 
    For example, a network proxy could intercept and forward these resources to an external detection service via its API.
    
    \item Anticipating that a specific malware detection system is consistently used for defense, the attacker swiftly generates a variant of the WebAssembly cryptojacking malware designed to evade detection at \step{4}.
    
    \item The attacker delivers the modified binary instead of the original one \step{5}, which initiates the cryptojacking process and compromises the browser \step{6}. The detection method is not capable of detecting the malicious nature of the binary, and the attack is successful.
    
\end{enumerate}


\msubsection{Methodology}

Our aim is to empirically validate the workflow in \autoref{fig:threat_model}, i.e., using Offensive Software Diversification in evading malware detection systems.
To achieve this, we employ WASM-MUTATE for generating \Wasm malware variants.
In this study, we categorize malware detection mechanisms as malware oracles, which can be of two types: binary and numeric. 
A binary oracle provides a binary decision, labeling a \Wasm binary as either malicious or benign. 
In contrast, a numeric oracle returns a numerical value representing the confidence level of the detection.

\begin{definition}{Malware oracle:}
    \label{malware_oracle_def}
    A malware oracle is a detection mechanism that returns either a binary decision or a numerical value indicating the confidence level of the detection.
\end{definition}


We employ VirusTotal as a numeric oracle and MINOS \cite{MINOS} as a binary oracle. 
VirusTotal is an online service that analyzes files and returns a confidence score in the form of the number of antivirus that flag the input file as malware, thus qualifying as a numeric oracle. 
MINOS, on the other hand, converts \Wasm binaries into grayscale images and employs a convolutional neural network for classification. 
It returns a binary decision, making it a binary oracle.


We use the wasmbench dataset \cite{Hilbig2021AnES} to establish a ground truth. 
After running the wasmbench dataset through VirusTotal and MINOS, we identify 33 binaries that are: 1) flagged as malicious by at least one VirusTotal vendor and, 2) are also detected by MINOS.
Then, to simulate the evasion scenario in \autoref{fig:threat_model}, we use WASM-MUTATE to generate \Wasm binary variants to evade malware detection (\step{4} in \autoref{fig:threat_model}).
We use WASM-MUTATE in two configurations: feedback-guided and stochastic diversification.

\begin{definition}{Feedback-guided Diversification:}
    \label{controlled_def}
    In feedback-guided diversification, the transformation process of a \Wasm program is guided by a numeric oracle, which influences the probability of each transformation. For instance, WASM-MUTATE can be configured to apply transformations that minimize the oracle's confidence score. Note that feedback-guided diversification needs a numeric oracle.
\end{definition}


\begin{definition}{Stochastic Diversification:}
    \label{uncontrolled_def}
    Unlike feedback-guided diversification, in stochastic diversification, each transformation has an equal likelihood of being applied to the input \Wasm binary.
\end{definition}


Based on the two types of malware oracles and diversification configurations, we examine three scenarios:
1) VirusTotal with a feedback-guided diversification, 2) VirusTotal with an stochastic diversification, and 3) MINOS with a stochastic diversification.
Notice that, the fourth scenario with MINOS and a feedback-guided diversification is not feasible, as MINOS is a binary oracle and cannot provide the numerical values required for feedback-guided diversification.

Our evaluation focuses on two key metrics: the success rate of evading detection mechanisms in VirusTotal and MINOS across the 33 flagged binaries, and the correctness of the generated variants.

\begin{definition}{Evasion rate:} This measures the efficacy of WASM-MUTATE in bypassing malware detection systems. 
    For each flagged binary, we input it into WASM-MUTATE, configured with the selected oracle and diversification strategy. 
    We then iteratively apply transformations to the output from the preceding step. 
    This iterative process is halted either when the binary is no longer flagged by the oracle or when a maximum of 1000 stacked transformations have been applied (see \autoref{stack_transform}).
    This process is repeated with 10 random seeds per binary to simulate 10 different evasion experiments per binary.
\end{definition}

\begin{definition}{Correctness:} This verifies the functional equivalence of the variants generated by WASM-MUTATE compared to the original binary. 
    We execute the variants that entirely evade VirusTotal, using controlled and stochastic diversification configurations with WASM-MUTATE for both metrics. 
    Our selection is limited to variants that allow us to fully reproduce the three components displayed in \autoref{fig:attack_crypto}. 
    We then gather the hashes generated by the cryptojacking binaries and their generation speed, comparing these hashes with those from the original binary. 
    If the hashes match, and the variant executes without error, with the minerpool component validating the hash, we can consider the variant as functionally equivalent.
\end{definition}

\msubsection{Results}

In \autoref{offensive:results:fast}, we present a comprehensive summary of the evasion experiments presented in \cite{EVASION}, focusing on two oracles: VirusTotal and MINOS\cite{MINOS}. 
The table is organized into two main categories to separate the results for each malware oracle. 
For VirusTotal, we further subdivide the results based on the two diversification configurations we employ: stochastic and feedback-guided diversification. 
In these subsections, the columns indicate the number of VirusTotal vendors that flag the original binary as malware (\#D), the maximum number of successfully evaded detectors (Max. \#evaded), and the average number of transformations required (Mean \#trans.) for each sample. 
We highlight in bold text the values for which the stochastic diversification or feedback-guided diversification setups best, the lower, the better.
The MINOS section solely includes a column that specifies the number of transformations needed for complete evasion. 
The table has 33 + 1 rows, each representing a unique \Wasm malware study subject. 
The final row offers the median number of transformations required for evasion across our evaluated setups and oracles. 

\newcolumntype{t}{>{\columncolor{Gray}}r}
\begin{table}
    \footnotesize
    \centering
    \begin{tabular}{c | r | r l | r l | t }
        \hline
        & \multicolumn{5}{|c|}{VirusTotal} & MINOS\cite{MINOS} \\
        \hline
        Hash & \#D & \multicolumn{2}{c|}{Stochastic diversification} & \multicolumn{2}{c|}{Feedback-guided diversification} & \\
        \hline
        &  & Max. #evaded & Mean #trans. & Max. #evaded & Mean #trans. & Mean #trans. \\
        \hline\hlined
        47d29959 &                 31 &             \textbf{26} &     N/A & 19 & N/A  & 100  \\ 
        9d30e7f0 &                 30 &             \textbf{24}  &      N/A & 17 & N/A & 419  \\ 
        8ebf4e44 &                 26 &             \textbf{21} &     N/A  & 13 & N/A & 92 \\
        \hline
        c11d82d &                 20 &       20        &  \textbf{355} & 20 & 446 & 115 \\ 
        0d996462 &                 19 &     19    &  \textbf{401} & 19 & 697 & 24 \\ 
        a32a6f4b &                 18 &       18       &  635 & 18 & \textbf{625} & 1 \\
        
        
        fbdd1efa &                 18 &         18      &  \textbf{310} & 18 & 726 & 1 \\ 
        d2141ff2 &                  9 &          9      &  \textbf{461} & 9 & 781 & 81 \\ 
        aafff587 &                  6 &          6      &  484 & 6 & \textbf{331} & 1 \\
        
        
        046dc081 &                  6 &          6      &  404 & 6 & \textbf{159} & 33 \\ 
        643116ff &                  6 &          6      &  \textbf{144} & 6 & 436 & 47 \\ 
        15b86a25 &                  4 &          4      &  253 & 4 & \textbf{131} & 1 \\
        
        
        
        006b2fb6 &                  4 &           4     &  \textbf{282} & 4 & 380 & 1\\ 
        942be4f7 &                  4 &           4     &  200 & 4 & 200 & 29\\ 
        7c36f462 &                  4 &           4     &  236 & 4 & \textbf{221} & 85\\
        
        
        fb15929f &                  4 &            4    &  \textbf{297} & 4 & 475 & 1 \\ 
        24aae13a &                  4 &         4       &  \textbf{252 } & 4 & 401 & 980\\ 
        000415b2 &                  3 &         3       &  302 & 3 & \textbf{34} & 960 \\
        
        4cbdbbb1 &                  3 &          3      &  295 & 3 & \textbf{72} & 1\\ 
        65debcbe &                  2 &          2      &  131  & 2 & \textbf{33} & 38 \\ 
        59955b4c &                  2 &          2      &  130  & 2 & \textbf{33} & 38 \\
        
        
        89a3645c &                  2 &           2     &  431 & 2 & \textbf{107} & 108\\
        a74a7cb8 &                  2 &           2     &  124 & 2 & \textbf{33} & 38 \\
        119c53eb &                  2 &           2     &  104 & 2 & \textbf{18} & 1 \\
        
        089dd312 &                  2 &           2     &  153 & 2 & \textbf{123} & 68\\
        c1be4071 &                  2 &           2     &  130 & 2 & \textbf{33} & 38\\
        dceaf65b &                  2 &           2     &  140 & 2 & 132 & 66\\
        
        6b8c7899 &                  2 &            2    &  143 & 2 & \textbf{33} & 38 \\
        a27b45ef &                  2 &         2       &  145 & 2 & \textbf{33} & 33\\
        68ca7c0e &                  2 &         2       &  137  & 2 & \textbf{33} & 38\\
        
        f0b24409 &                  2 &         2       &  127  & 2 & \textbf{11} & 33 \\
        5bc53343 &                  2 &         2       &  118  & 2 & \textbf{33} & 33 \\
        e09c32c5 &                  1 &         1       &  \textbf{120}  & 1 & 488 & 15 \\
        \hline\hline
        Median &                         &         &      218  &   & 131 & 38
    \end{tabular}
    \caption{
        The table has two main categories for each malware oracle, corresponding to the two oracles we use: VirusTotal and MINOS. 
        For VirusTotal, divide the results based on the two diversification configurations: stochastic and feedback-guided diversification. 
        We provide columns that indicate the number of VirusTotal vendors that flag the original binary as malware (\#D), the maximum number of successfully evaded detectors (Max. \#evaded), and the average number of transformations required (Mean \#trans.) for each sample. 
        We highlight in bold text the values for which diversification setups are best, the lower, the better.
        The MINOS section includes a column that specifies the number of transformations needed for complete evasion. 
        The final row offers the median number of transformations required for evasion across our evaluated setups and oracles. 
    }
    \label{offensive:results:fast}
\end{table}

\begin{strategy}[Stochastic diversification to evade VirusTotal]
    \label{stochastic_div_vt}
    We execute a stochastic diversification with WASM-MUTATE, setting a limit of 1000 iterations for each binary. 
    In every iteration, we query VirusTotal to determine if the newly generated binary can elude detection. 
    We repeat this procedure with ten distinct seeds for each binary, replicating ten different evasion experiments. 
    As the stochastic diversification section of \autoref{offensive:results:fast} illustrates, we successfully produce variants that fully evade detection for 30 out of 33 binaries. 
    The average amount of iterations required to produce a variant that evades all detectors oscillates between 120 to 635 stacked transformations. 
    The mean number of iterations needed never exceeds 1000 stacked transformations. 
    However, three binaries remain detectable under the stochastic diversification setup. 
    In these instances, the algorithm fails to evade 5 out of 31, 6 out of 30, and 5 out of 26 detectors. 
    This shortfall can be attributed to the maximum number of iterations, 1000, that we employ in our experiments. 
    Increasing iterations further, however, seems unrealistic. 
    If certain transformations enlarge the binary size, a significantly large binary could become impractical due to bandwidth limitations. 
    In summary, stochastic diversification with WASM-MUTATE markedly reduces the detection rate by VirusTotal antivirus vendors for cryptojacking malware, achieving total evasion in 30 out of 33 (90\%) cases within the malware dataset. 
    %WASM-MUTATE proves capable of successfully evading detection systems in just a few minutes.    
\end{strategy}
    

\begin{strategy}[Feedback-guided diversification to evade VirusTotal]
    \label{guided_div_vt}
    stochastic diversification does not guide the diversification based on the number of evaded detectors, it is purely random, and has some drawbacks.
    For example, some transformations might suppress other transformations previously applied.
    We have observed that, by carefully selecting the order and type of transformations applied, it is possible to evade detection systems in fewer iterations.
    This can be appreciated in the results of the feedback-guided diversification part of \autoref{offensive:results:fast}.
    The feedback-guided diversification setup successfully generates variants that totally evade the detection for 30 out of 33 binaries, it is thus as good as the stochastic setup.
    Remarkably, for 21 binaries out of 30, feedback-guided needs only 40\% of the calls the stochastic diversification setup needs, demonstrating larger efficiency. 

    %In practice, a potential attacker may be limited by budget on the number of transformations applicable to the malware binary, e.g., the number of queries to VirusTotal. 
    %Additionally, the performance of the resulting binary benefits from this approach.

\end{strategy}
  
\begin{strategy}[Stochastic diversification to evade MINOS]
    \label{stochastic_div_minos}
    Relying exclusively on VirusTotal for detection could pose issues, particularly given the existence of specialized solutions for \Wasm, which differ from the general-purpose vendors within VirusTotal. 
    In \autoref{background:wasm:analysis} we highlight several examples of such solutions.
    Yet, for its simplicity, we extend this experiment by using MINOS\cite{MINOS}, an antivirus specifically designed for \Wasm. 
    The results of evading MINOS can be seen in the final column of \autoref{offensive:results:fast}.
    The bottom row of \autoref{offensive:results:fast} highlights that fewer iterations are required to evade MINOS than VirusTotal through WebAssembly diversification, indicating a greater ease in eluding MINOS.
    The stochastic diversification setup requires a median iteration count of 218 to evade VirusTotal. 
    In contrast, the feedback-guided diversification setup necessitates only 131 iterations. 
    Remarkably, a mere 38 iterations are needed for MINOS. 
    WASM-MUTATE evaded detection for 8 out of 33 binaries in a single iteration. 
    This result implies a vulnerability in the MINOS model to binary diversification.
\end{strategy}
    
%\msubsection{Efficiency and correctness results}
\vspace{5mm}
\begin{strategy}[\Wasm variants correctness]
    \label{evasion_impact}
    To evaluate the correctness of the malware variants created with WASM-MUTATE, we focused on six binaries that we could build and execute end-to-end, as these had all three components outlined in \autoref{fig:attack_crypto}. 
    We select only six binaries because the process of building and executing the binaries involves three components: the \Wasm binary, its JavaScript complement, and the miner pool. 
    These components were not found for the remaining 24 evaded binaries in the study subjects.
    For the six binaries, we then replace the original WebAssembly code with variants generated using VirusTotal as the malware oracle and WASM-MUTATE for both controlled and stochastic diversification configurations. 
    We then execute both the original and the generated variants. 
    We assess the correctness of the variants by examining the hashes they generate.
    Our findings show that all variants generated with WASM-MUTATE are correct, i.e., they generate the correct hashes and execute without error.
    Additionally, we found that 19\% of the generated variants surpassed the original cryptojacking binaries in performance.
\end{strategy}

% what's missing is to take a step back: what do those results validate? Wasm-mutate? the evasion process? both? what does this mean wrt to diversification in general?

%\todo{Do those results hold only for Spectre? do they generalize to other attacks? do you recommend binary diversification for any Wasm build pipeline as an hardening technique? how far are we from that?}

\begin{tcolorbox}[title=Reflection,boxrule=1pt,arc=.2em,boxsep=1.0mm]
    Our experiments conclusively demonstrate that WASM-MUTATE can effectively circumvent malware detection systems. 
    A possible key factor behind this is a misguided perception of resilience. 
    Malware detection is a well-known difficult problem \cite{cohen1987computer}. 
    Yet, prior research on static WebAssembly malware detection has shown an erroneous presumption: the existing of only metadata (\Wasm custom sections) obfuscation, or the complete absence of obfuscation techniques for \Wasm \cite{Minesweeper, MinerRay, SEISMIC, RAPID, MINOS}. 
    As explored in \autoref{sota:sw}, a software diversification engine can potentially function as an obfuscator. %%ORIGINAL
    The discussed use case partially demonstrates the assumption of non-existing obfuscators might be incorrect. 
    Consequently, our software diversification tools provide a viable solution for enhancing the accuracy of \Wasm malware detection systems.

    %Ultimately, our findings suggest that feedback-guided diversification is more efficient than stochastic diversification, requiring fewer transformations to avoid detection.    
\end{tcolorbox}


\begin{tcolorbox}[title=Contribution paper,boxrule=1pt,arc=.2em,boxsep=1.0mm]
    WASM-MUTATE generates correct and performant variants of WebAssembly cryptojacking that successfully evade malware detection.
    The case discussed in this section is fully detailed in Cabrera-Arteaga \etal "WebAssembly Diversification for Malware Evasion"
    \emph{at Computers \& Security, 2023}
    \url{https://www.sciencedirect.com/science/article/pii/S0167404823002067}. 
\end{tcolorbox}


% Efficiency
%We have found that 
%This improvement is attributed to WASM-MUTATE's ability to introduce code optimizations. 
%Additionally, debloating transformations, which eliminate unnecessary structures and dead code, resulted in a higher hash generation rate during the initial seconds of mining, likely due to faster compilation times. 
%This suggests that focused optimization serves as a valuable tool for evasion in browsers.
% The contrary case.
%On the contrary, 80\% of the generated variants are less efficient than the original binary, with the least efficient variant operating at only 20\% of the original hash generation rate. 
%This performance drop is primarily due to non-optimal transformations introduced by WASM-MUTATE. 
%Variants generated through stochastic diversification are generally slower.
%In summary, feedback-guided diversification yielded variants that evaded VirusTotal detection with minimal performance overhead—the worst-performing variant was only 1.93 times slower than the original.